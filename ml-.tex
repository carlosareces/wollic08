%\begin{defn}[Syntax]\label{syntax}
%Let $\prop=\{p_1, p_2, \dots\}$ (the \textit{propositional symbols})
%and $\rel=\{r_1, r_2, \dots\}$ (the \textit{relational symbols}) be
%pairwise disjoint, countable infinite sets of symbols. The set
%$\forms$ of formulas of $\tlm$ in the signature $\langle
%\prop,\rel\rangle$ is defined as:
%$$
%\forms ::= p \mid \known \mid \lnot \varphi \mid \varphi_1 \land
%\varphi_2 \mid \diam{R} \varphi \mid \remember \varphi,
%$$
%where $p \in \prop$, $R \in \rel$  and $\varphi, \varphi_1,
%\varphi_2 \in \forms$.
%\end{defn}



In this section we study another member of the memory logics family,
called $\tlm$, in which the behavior of the \emph{remember} operator
is highly coupled with the modal transitions. In this logic, every
time we make a modal step, we are constrained to remember the
current state. We change the semantic definition of $\diam{r}$ to
be:
$$
\begin{array}{rcl}
\diam{W,\rels,V,S}, w \models \diam{r} \varphi & \iff & \exists w' \in W, R_r(w,w') \textrm{ and } \\
& & \diam{W,\rels,V, S \cup \{w\}}, w' \models \varphi
\end{array}
$$
We will show how this change in the semantic of $\diam{r}$ leads to
some interesting results.

We define the \textit{Ehrenfucht game} $\game$ for $\tlm$ in the
same way as for $\tl$, with the following modifications: in step 2,
Spoiler always remembers the current world, and the game goes on
with $E(\model_1[w_1], \model_2[w_2], v_1, v_2)$ after Duplicator
response.

%We define bisimulation and $n$-bisimulation between models
%($\model_1 \bisim \model_2$ and $\model_1 \bisim_n \model_2$) in the
%same way as before. In particular, note that Theorem~\ref{bisim} and
%\ref{thm:hennesy} hold for $\tlm$.
%
%SACAR LOS TEOREMAS Y SOLO HACER REFERENCIA A LOS QUE YA TENEMOS
%
%\begin{thm}
%Let $\model_1, \model_2$ be two models, $w_1 \in \model_1, w_2 \in
%\model_2$. If $w_1 \bisim w_2$ then $w_1 \leftrightsquigarrow w_2$.
%\end{thm}
%
%\begin{thm}
%Let $\model_1, \model_2$ be two image finite models. Then for every
%$w_1 \in \model_1$ and $w_2 \in \model_2$,  $w_1 \bisim w_2$ iff
%$w_1 \leftrightsquigarrow w_2$.
%\end{thm}


%\begin{defn}[$n$-Bisimulation]
%We say that two models $\model_1$ and $\model_2$ are $n$-bisimilar
%($\model_1 \bisim_n \model_2$) when there exists $w_1 \in \model_1$
%and $w_2 \in \model_2$ such that they agree in the same
%propositional symbols and Duplicator has a winning strategy on
%$E_n(\model_1, \model_2, w_1, w_2)$. In this case we will also say
%that $w_1$  and $w_2$ are $n$-bisimilar ($\model_1, w_1 \bisim_n
%\model_2, w_2$).
%\end{defn}
%
%FALTA DEFINIR $E_n$ Y $EQUIV_n$
%
%\begin{thm}\label{thm:bisim-n-to-modal-n}
%Let $\model_1,\model_2$ be two models, $w_1 \in \model_1, w_2 \in
%\model_2$. If $w_1 \bisim_n w_2$ then $w_1 \leftrightsquigarrow_n
%w_2$.
%\end{thm}
%
%\begin{pf}
%The proof can be done by induction on $n$.
%\end{pf}
%
%For a natural number $k$, the restriction of a model $\model$ to $k$
%(notation: $\model \upharpoonright k$) is defined as the submodel
%containing only states whose height is at most $k$.
%
%
%\begin{thm}\label{thm:bisim-n-to-prune}
%Let $\model$ be a model and $w \in \model$. For every $n \geq 0$,
%$\model, w \bisim_n (\model \upharpoonright n), w$.
%\end{thm}
%
%\begin{pf}
%It is easy to see that in an $n$-round game the players only visit
%states with height at most $n$ and thus, as far as the players are
%concerned, $\model$ and $\model\upharpoonright n$ are the same.
%Therefore Duplicator can always copy Spoiler's moves.
%\end{pf}
%
%
%% \begin{pf}
%% We follow the same idea as in Theorem~\ref{bisim}. The only non
%% trivial case is $\varphi = \diam{R} \psi$. We prove that
%% $\model_1,w_1 \models \diam{R} \psi$ implies $\model_2,w_2 \models
%% \diam{R} \psi$. Suppose $\model_1,w_1 \models \diam{R} \psi$. Then
%% there is $v_1\in \model_1$ such that $w_1R_1v_1$ and
%% $\model_1[w_1],v_1\models\psi$. Suppose that Spoiler chooses model
%% $\model_1$ at step 1 and $v_1$ as an $R_1$-successor. By
%% Lemma~\ref{lem:strategy}, Duplicator has a winning winning strategy
%% on $E(\model_1[w_1], \model_2[w_2], w_1, w_2)$, and following that
%% strategy, he finds $v_2$ such that $\props(v_1)=\props(v_2)$,
%% $w_2R_2v_2$, and keeps a winning strategy on $E(\model_1[w_1],
%% \model_2[w_2], v_1, v_2)$. By inductive hypothesis and the fact that
%% $\model_1[w_1],v_1\models\psi$, we conclude
%% $\model_2[w_2],v_2\models\psi$ and therefore
%% $\model_2,w_2\models\diam{R} \psi$. The other direction is
%% identical.
%% \end{pf}


We now give some results related to the expressive power of $\tlm$.

\begin{thm}
$\bml$ over the signature $\diam{\prop \cup \{known\}, \rel}$ is
strictly less expressive than $\tlm$ over the signature
$\diam{\prop, \rel}$.
\end{thm}

\begin{pf}
Exactly the same proof used to prove that $\bml < \tl$.
\end{pf}

\begin{thm}
$\tlm < \tl$.
\end{thm}

\begin{pf}
It is easy to see that there is a translation from $\tlm$ to $\tl$
which maps $\diam{R}\varphi$  to  $\remember\diam{R}\varphi$. To see
that the inclusion is strict, let $\model_1=\diam{\{w,v,r\}, R_1,
V_1, \emptyset}$ and $\model_2=\diam{\{w,v,r\}, R_2, V_2,
\emptyset}$ such that $R_1=\{(w,v),(v,r),(r,w)\}$,
$R_2=\{(w,v),(v,r),(r,v)\}$ and $V_1(p) = V_2(p) = \emptyset$ for
all $p \in \prop$.
% \begin{center}
% \begin{pgfpicture}{0cm}{0cm}{4cm}{1cm}
%     \pgfnodecircle{nodo1}[stroke]{\pgfxy(0.5,0.5)}{0.25cm}
%     \pgfnodecircle{nodo2}[stroke]{\pgfxy(2.0,0.5)}{0.25cm}
%     \pgfnodecircle{nodo3}[stroke]{\pgfxy(3.5,0.5)}{0.25cm}
%
%     \pgfsetlinewidth{0.4pt}
%     \pgfnodesetsepstart{1.5pt}
%     \pgfnodesetsepend{3pt}
%     \pgfsetendarrow{\pgfarrowtriangle{4pt}}
%     \pgfnodeconnline{nodo1}{nodo2}
%     \pgfnodeconnline{nodo2}{nodo3}
% %    \pgfnodeconncurve{nodo3}{nodo1}{225}{315}{1cm}{1cm}
%
%     \pgfnodelabel{nodo1}{nodo1}[-17][0pt]{\pgfbox[center,base]{$w$}}
%     \pgfnodelabel{nodo3}{nodo3}[-30][1.5cm]{\pgfbox[center,base]{$\model_1$}}
% \end{pgfpicture}
% \hspace{20mm}
% \begin{pgfpicture}{0cm}{0cm}{4cm}{1cm}
%     \pgfnodecircle{nodo1}[stroke]{\pgfxy(0.5,0.5)}{0.25cm}
%     \pgfnodecircle{nodo2}[stroke]{\pgfxy(2.0,0.5)}{0.25cm}
%     \pgfnodecircle{nodo3}[stroke]{\pgfxy(3.5,0.5)}{0.25cm}
%
%     \pgfsetlinewidth{0.4pt}
%     \pgfnodesetsepstart{1.5pt}
%     \pgfnodesetsepend{3pt}
%     \pgfsetendarrow{\pgfarrowtriangle{4pt}}
%     \pgfnodeconnline{nodo1}{nodo2}
%     \pgfnodeconnline{nodo2}{nodo3}
% %    \pgfnodeconncurve{nodo3}{nodo2}{225}{315}{0.2cm}{1cm}
%
%     \pgfnodelabel{nodo1}{nodo1}[-17][0pt]{\pgfbox[center,base]{$w$}}
%     \pgfnodelabel{nodo3}{nodo3}[-30][1.5cm]{\pgfbox[center,base]{$\model_2$}}
% \end{pgfpicture}
% \vspace{8mm}
% \end{center}
We prove that $\model_1$ and $\model_2$ are $\tlm$-bisimilar. As
every state in both models has a unique successor, Duplicator has
only one way of playing, and this only way is indeed a winning
strategy. From this follows that no formula in $\tlm$ can
distinguish $\model_1$ and $\model_2$. On the other hand, let
$\varphi = \diam{R}\remember\diam{R}\diam{R}\known$ be a
$\tl$-formula. It is easy to see that $\model_1,w \not \models
\varphi$, but $\model_2, w \models \varphi$.
\end{pf}

We now establish an equivalence result between $\bml$ and $\tlm$ for
the class of tree-like models. We first define, given a formula
$\varphi$, a new formula $\varphi^\sharp$, as the result of
replacing all the ocurrences of $\known$ that are in $\varphi$ at
modal depth zero by $\top$. More formally:

\begin{displaymath}\label{sharp}
\begin{array}{rcl}
p^\sharp & = & p \quad p \in \prop\\
\known^\sharp & = & \top \\
(\lnot \varphi)^\sharp & = & \lnot \varphi^\sharp \\
(\varphi_1 \land \varphi_2)^\sharp & = & \varphi_1^\sharp \land \varphi_2^\sharp \\
(\remember \varphi)^\sharp & = & \remember \varphi^\sharp\\
(\diam{R} \varphi)^\sharp & = & \diam{R} \varphi
\end{array}
\end{displaymath}


\begin{lem}\label{lem:replace}
$\model, w \models \remember \varphi$ iff $\model, w \models
\varphi^\sharp$.
\end{lem}

\begin{pf}
We proceed by induction. The case for $\known$, the propositional
symbols and booleans are straightforward. We analyze the other
cases:
\begin{itemize}
 \item $\varphi = \remember \psi$. $\model, w \models \remember\remember\psi$ iff $\model, w \models \remember \psi$ iff (by inductive hypothesis) $\model, w \models \psi^\sharp$ iff $\model, w \models (\psi^\sharp)^\sharp$ iff (by inductive hypothesis) $\model, w \models \remember (\psi^\sharp)$ iff $\model, w \models (\remember\psi)^\sharp$
\item $\varphi = \diam{R} \psi$. $\model, w \models \remember \diam{R} \psi$ iff (by definition) $\model[w], w \models \diam{R} \psi$ iff (by inductive hypothesis) $\model[w], w \models (\diam{R} \psi)^\sharp$ iff (by definition) $\model, w \models (\diam{R} \psi)^\sharp$
\end{itemize}
\end{pf}

We are now ready to define an equivalence preserving translation for
the tree-like models.

\begin{defn}[Translation]\label{def:tr-tlm-k}
Let $known$ be a new proposition symbol not in $\prop$. The
translation $\Tr$, taking $\tlm$ formulas over the signature
$\diam{\prop, \rel}$ to $\bml$-formulas over the signature
$\diam{\prop \cup \{known\}, \rel}$ is defined as:

\begin{displaymath}
\begin{array}{rcl}
\Tr(p) & = & p \quad p \in \prop\\
\Tr(\known) & = & known \\
\Tr(\lnot \varphi) & = & \lnot \Tr(\varphi) \\
\Tr(\varphi_1 \land \varphi_2) & = & \Tr(\varphi_1) \land \Tr(\varphi_2) \\
\Tr(\diam{R} \varphi) & = & \diam{R} \Tr(\varphi) \\
\Tr(\remember \varphi) & = & \Tr(\varphi^\sharp)
\end{array}
\end{displaymath}

\end{defn}

Taking the same model equivalence used to compare $\bml$ with $\tl$,
the following result can be established:

\begin{pro}[Satisfiability preserving]\label{prop:sat-preserv-tree}
Let $\model$ be a a tree-like model. Then
$\model,w\models_{\tlm}\varphi$ iff $\model,w\models_{\bml}
\Tr(\varphi)$.
\end{pro}

\begin{pf}
We proceed by induction on $\varphi$. The propositional and boolean
cases are trivial. The $\known$ case is also easy given the
definitions. For the diamond case. Given that $\model$ is tree-like,
the remember operator has no effect beyond modal operators, so
$\model, w \models_{\tlm} \diam{R}\psi$ iff exists $v$ such that
$wRv$ and $\model,v\models_{\tlm} \psi$. By inductive hypothesis,
$\model,v\models_{\tlm} \psi$ iff $\model, v \models_{\bml}
\Tr(\psi)$, and by definition $\model, w \models_{\bml} \diam{R}
\Tr(\psi)$. Finally, let's see the case for remember. By
Lemma~\ref{lem:replace}, $\model, w \models_{\tlm} \remember \psi$
iff $\model, w \models_{\tlm} \psi^\sharp$. By inductive hypothesis,
$\model, w \models_{\bml} \Tr(\psi^\sharp)$.
\end{pf}



%\subsection{Finite model property}

% \begin{lem}\label{lem:frontera-k}
% Let $\model$ be a model and $w \in \model$. Then $(\model
% \upharpoonright n),w \bisim_n \model, w$.
% \end{lem}
%
% \begin{pf}
% In the game $E_n(\model, \model \upharpoonright n, w, w)$ the
% winning strategy for Duplicator consists in copying Spoiler moves.
% This can always be done because Spoiler cannot go beyond $n$ steps
% from $w$.
% \end{pf}

% \begin{thm}
% Let $\model$ be a model and $w \in \model$. For every $n \geq 0$,
% $\model, w \leftrightsquigarrow_n (\model \upharpoonright n), w$.
% \end{thm}
%
% \begin{pf}
% This result follows from Theorem~\ref{thm:bisim-n-to-prune} and
% Theorem~\ref{thm:bisim-tlm-n-to-modal-n}.
% \end{pf}



\begin{pro}[Tree model
property]\label{prop:tree-model-property} For every $\tlm$-model
$\model$ and $w\in\model$ there is a tree-like model $\model'$ where
$w\in\model'$ and $\model,w\bisim\model',w$.
\end{pro}

\begin{pf}
Let $\model=\diam{W,R,V,S}$.  We prove the result for the unimodal
case, the generalization to the multimodal case is straightforward.
We define the model $\model'=\diam{W',R',V',S'}$ as follows. Its
domain $W'$ consists on all finite sequences $(u_0,\dots,u_n)$ such
that $u_0=w$, $n\geq 0$ and there is a path $u_0Ru_1\dots Ru_n$ in
$\model$. Define $(u_0,\dots,u_n)R(v_0,\dots,v_m)$ to hold if
$m=n+1$, $u_i=v_1$ for $i=0,\dots,n$ and $u_nRv_m$ holds in
$\model$. The valuation $V'$ is defined by setting
$(u_0,\dots,u_n)\in V'(p)$ iff $u_n\in V(p)$. Finally,
$(u_0,\dots,u_{n-1},u_n)\in S'$ iff $u_n\in\{u_0,\dots,u_{n-1}\}$ or
$u_n\in S$.

We call $s_i$ the sequence $(v_0,\dots,v_i)$ for every
$i=0,\dots,n$. Let us see that Duplicator has a winning strategy in
the game $E(\model,\model',w,w)$. It is sufficient to see that in
the game
$E(\model[v_0,\dots,v_n],\model'[s_0,\dots,s_n],v_{n+1},s_{n+1})$,
Duplicator can always answer successfully to Spoiler's move.
\begin{itemize}
\item
If Spoiler chooses $\model[v_0,\dots,v_n]$ and some $v_{n+1}$, a
successor of $v_n$, Duplicator chooses the sequence
$s_{n+1}=s_nv_{n+1}$.

\item
If Spoiler chooses $\model'[s_0,\dots,s_n]$ and $s_{n+1}=s_nv_{n+1}$
(for some $v_{n+1}$), a successor of $s_n$, Duplicator chooses the
state $v_{n+1}$.
\end{itemize}
By definition, it is easy to see that $s_{n+1}$ and $v_{n+1}$ agree.
Observe that at this stage of the game the current ``memory" of
$\model[v_0,\dots,v_n]$ is $S \cup \{v_0,\dots,v_n\}$ and the
current ``memory" of $\model'[s_0,\dots,s_n]$ is $S' \cup
\{s_0,\dots,s_n\}$. Then it is also clear that $v_{n+1}$ was
remembered at this stage of the game, if and only if $s_{n+1}$ also
was. More formally, on the one hand, $v_{n+1}\in S \cup
\{v_0,\dots,v_n\}$ implies $s_{n+1} \in S'$ by definition. On the
other hand if $s_{n+1}\in S' \cup \{s_0,\dots,s_n\}$, we have
$s_{n+1}\in S'$ (since there are no cycles in $\model'$) and by
definition $v_{n+1}\in S \cup \{v_0,\dots,v_n\}$.
\end{pf}

If $\model$ is a $\tlm$-model, we denote with $\model_T$ the
tree-like model constructed in the above proposition.

%\begin{lem}\label{lem:bisim_bml_then_bisim_tlm}
%Let $\model,\nodel$ be two $\tlm$ tree-like models. For any $n$, if
%the equivalent $\bml$-models satisfy
%$\model,w\bisim_n^{\bml}\nodel,v$ then
%$\model,w\bisim_n^{\tlm}\nodel,v$.
%\end{lem}
%
%\begin{pf}
%Let $\model=\diam{W_1,R_1,V_1,S_1}$ and
%$\nodel=\diam{W_2,R_2,V_2,S_2}$.
%%
%Recall the definition of $n$-bisimulation for $\bml$
%\cite[Definition 2.30]{BRV01}, involving the relations
%$Z_0\supseteq\dots\supseteq Z_{n}$. We define a winning strategy for
%Duplicator in terms of $Z_0,\dots,Z_{n}$.
%
%We start in the game $E(\model,\nodel,w,v)$. We show that if
%$E(\model',\nodel',w',v')$ is some $i$th stage of that game ($i\leq
%n$) and $w'Z_{n-i}v'$, then for any choice $u$ of Spoiler,
%Duplicator can always answer with $t$ such that $uZ_{n-i-1}t$.
%Suppose Spoiler chooses $u\in\model'$ such that $w'Ru$. By
%definition of $Z_{n-i}$, there exists $t\in\nodel'$ such that $v'Rt$
%and $vZ_{n-i-1}t$. It follows from the definition of $Z_0$ that $u$
%and $t$ agree. Observe that the game $E(\model',\nodel',w',v')$ is
%just equivalent to $E(\model,\nodel,w',v')$, since both $\model$ and
%$\nodel$ are acyclic. Therefore $u\in S_1$ iff $t\in S_2$. The case
%Spoiler chooses to play in $\nodel'$ is similar and this completes
%the proof.
%\end{pf}

%\begin{thm}\label{thm:fmp-tlm}
%Let $\model$ be a model. For every $n$ there is a finite tree-like
%model $\nodel$ such that $\model\bisim_n\nodel$.
%\end{thm}

%\begin{pf}
%By Proposition~\ref{prop:tree-model-property},
%$\model\bisim^{\tlm}\model_T$. By
%Theorem~\ref{thm:bisim-n-to-prune},
%$\model_T\bisim_n^{\tlm}\model_T\upharpoonright n$. Applying the
%selection process in \cite[Theorem 2.34]{BRV01} on the equivalent
%$\bml$-model for $\model_T\upharpoonright n$ we obtain  a finite
%tree-like model $\nodel$ such that $\model_T\upharpoonright
%n\bisim_n^{\bml}\nodel$. Applying
%Lemma~\ref{lem:bisim_bml_then_bisim_tlm}, we know that
%$\model_T\upharpoonright n\bisim_n^{\tlm}\nodel$ and therefore
%$\model \bisim_n^{\tlm}\nodel$.
%\end{pf}

\begin{thm}
If $\varphi\in\tlm$ is satisfiable then it is satisfiable on a
finite model.
\end{thm}

\begin{pf}
Let $\varphi$ be a $\tlm$ formula, and suppose $\model,w \models
\varphi$. By Proposition~\ref{prop:tree-model-property}, there is a
tree-like model $\model'$ such that $\model',w \models \varphi$.
Using Proposition~\ref{prop:sat-preserv-tree}, we can turn to basic
modal logic $\bml$, and assert that $\model',w \models_{\bml}
Tr(\varphi)$. Now we can use the finite tree model property for
basic modal logic \cite{??}, so there must be a finite tree-like
model $\model''$ such that $\model'',w \models_{\bml} Tr(\varphi)$.
Finally, we can use Proposition~\ref{prop:sat-preserv-tree} again,
and conclude $\model'',w \models \varphi$.
\end{pf}


%\begin{cor}[Finite model property]
%If $\varphi\in\tlm$ is satisfiable then it is satisfiable on a
%finite model.
%\end{cor}

%\begin{thm}[Alternative proof for the finite model property] (without using Lemma~\ref{lem:bisim_bml_then_bisim_tlm})
%If $\varphi\in\tlm$ is satisfiable then it is satisfiable on a
%finite model.
%\end{thm}
%\begin{pf}
%Let $\varphi$ be a $\tlm$ formula, and suppose $\model,w \models
%\varphi$. By Proposition~\ref{prop:tree-model-property}, there is a
%tree-like model $\model'$ such that $\model',w \models \varphi$.
%Using Proposition~\ref{prop:sat-preserv-tree}, we can turn to basic
%modal logic $\bml$, and assert that $\model',w \models_{\bml}
%Tr(\varphi)$. Now we can use the finite tree model property for
%basic modal logic \cite{??}, so there must be a finite tree-like
%model $\model''$ such that $\model'',w \models_{\bml} Tr(\varphi)$.
%Finally, we can use Proposition~\ref{prop:sat-preserv-tree} again,
%and conclude $\model'',w \models \varphi$.
%\end{pf}


\begin{cor}[Decidability]\label{cor:tlm-decidability}
$\tlm$ is decidable.
\end{cor}

\begin{pf}
Given $\varphi$ with modal degree $k$, it suffices to observe that
in the selection process (in the proof of
Theorem~\ref{thm:fmp-tlm}), one can bound the number of elements of
the constructed model by $n^k$, where $n$ is the number of
non-equivalent formulas of degree at most $k$, containing the same
propositional symbols of $\varphi$. Therefore, to decide the
satisfiability of $\varphi$ one can check if $\model\models\varphi$
for all models $\model$ with size at most $n^k$.
\end{pf}

\begin{thm}[\mbox{\rm\textsc{pspace}}-complete]
The satisfaction problem for $\tlm$ is \textsc{pspace}-complete.
\end{thm}
\begin{pf}
The lower \textsc{pspace}-hard bound for $\tlm$ follows from the
lower bound for the basic modal logic~\cite{BRV01}. We show a
matching upper bound by reducing the problem to the satisfiability
problem for basic modal logic. Given a $\tlm$-formula $\varphi$, we
compute $\Tr(\varphi)$ using Definition~\ref{def:tr-tlm-k}. Note
that the size of the translated formula $\Tr(\varphi)$ is actually
smaller than $\varphi$. Let suppose that $\Tr(\varphi)$ has a model,
that is $\model, w \models_{\bml} \Tr(\varphi)$. This happens iff
there exists a tree-like model $\model'$ such that $\model', w
\models_{\bml} \Tr(\varphi)$ (by the tree model property of $\bml$)
iff $\model', w \models_{\tlm} \varphi$ (by
Proposition~\ref{prop:sat-preserv-tree}) iff there exists a model
$\nodel$ such that $\nodel,w \models_{\tlm} \varphi$ (by the tree
model property for $\tlm$, stated in
Proposition~\ref{prop:tree-model-property}). Therefore, given
$\varphi$, determining the $\bml$-satisfiability of $\Tr(\varphi)$
(and we can do this with a \textsc{pspace} algorithm~\cite{BRV01})
is enough to decide the $\tlm$-satisfiability of $\varphi$.
\end{pf}


%\subsection{Undecidable again}

No we want to analyze the case of $\tlem$ with respect to
decidability. We first prove that $\tlem$ lack the finite model
property.

\begin{thm}\label{thm:infinite_model}
There is a formula $\textit{Inf} \in \tlem$ such that $\mathcal{M},w
\models \textit{Inf}$ implies that the domain of $\mathcal{M}$ is an
infinite set.
\end{thm}

\begin{pf}
Consider the following formulas:
$$
\begin{array}{rl}
(\textit{Back}) & p \land [R] \lnot p  \land \diam{R} \top \land [R]\diam{R} (\known \land p) \\
&$``The root is irreflexive, and in symmetric relation with all its (nontempty) successors.''$\\
(\textit{Spy}) & [R][R]( \lnot p \to \diam{R} (\known \land p \land \diam{R} ( \known \land \lnot \diam{R} ( \known \land \lnot p))))\\
&$``A 2-step successor is a 1-step successor, with a non-symmetric relation between them.''$\\
(\textit{Irr}) & \lnot \diam{R} ( \lnot p \land \diam{R} ( \lnot p \land \known)))\\
&$``All successors are irreflexive.''$\\
(\textit{Succ}) & [R]\diam{R} \lnot p \\
&$``All successors have a successor different from the root.''$\\
(\textit{3cyc}) & \lnot \diam{R}   \diam{R} ( \lnot \known \land \diam{R} ( \lnot \known \land \diam{R} ( \known \land \lnot p)))\\
&$``There are no cycles of 3 elements between successors of the root.''$\\
(\textit{Tran}) & [R] ( \lnot p \to \diam{R} ( \known \land p \land [R] ( \lnot p \land \lnot \known \to \diam{R}( \known \land p \land \diam{R} ( \known \land \lnot p \land  \diam{R}(\known \land \lnot p) )))))\\
&$``Every pair of successors $u$ and $v$ are related (either $uRv$ or $vRu$).''$\\
\end{array}
$$

Let $\textit{Inf}$ be $\textit{Back} \land \textit{Spy} \land
\textit{Irr} \land \textit{Succ} \land \textit{3cyc} \land
\textit{Tran}$. Let $\model=\diam{W, R, V, \emptyset}$. One can
follow the idea used in the proof of
Theorem~\ref{thm:infinite_model} to show that if $\model, w \models
\textit{Inf}$, then $W$ is infinite.
%
%Suppose $\model, w \models \textit{Inf}$. Let $B = \{b \in W \mid
%wRb\}$. Because $\textit{Back}$ is satisfied, $w \not \in B$, $B
%\not= \emptyset$ and for all $b \in B$, $bRw$. Because
%$\textit{Spy}$ is satisfied, if $a \not= w$ and $a$ is a successor
%of an element of $B$ then $a$ is also an element of $B$. As
%$\textit{Irr}$ is satisfied at $w$, every state in $B$ is
%irreflexive. As $\textit{Succ}$ is satisfied at $w$, every point in
%$B$ has a successor distinct from $w$. As $\textit{3cyc}$ is
%satisfied, there cannot be $3$ different elements in $B$ forming a
%cycle, and this sentence together with $\textit{Tran}$ force $R$ to
%transitively order $B$.
%
%It follows that $B$ is an unbounded strict partial order, hence
%infinite, and so is $W$.
\end{pf}

We now show that the satisfiability problem for $\tlem$ is
undecidable, by encoding the $\omega \times \omega$ tiling problem.

\begin{thm}
The satisfiability problem for $\tlem$ is undecidable.
\end{thm}
\begin{pf}
Let $T=\{T_1,\dots,T_n\}$ be a set of tile types. Given a tile type
$T_i$, $u(T_i)$, $r(T_i)$, $d(T_i)$, $l(T_i)$ will represent the
colors of the up, right, down and left edges of $T_i$ respectively.
Define:

\begin{displaymath}
\begin{array}{rl}
(\textit{Back}) & p \land [S]\lnot p \land \diam{S} \top \land [S]\diam{S}(\known \land p) \land [S][S](\known \land p)\\
(\textit{Spy}) & [S][\dag](\lnot p \land \diam{S}(\known \land p \land \diam{S}(\known \land \lnot\diam{\dag}\known))) \quad \dag \in \{R,U\}\\
(\textit{Grid}) & [S]\diam{\dag} \top \quad \dag \in \{R,U\}\\
(\textit{Func}) & [S][\dag]\diam{S}\diam{S}(\known \land \diam{\dag}\known \land [\dag]\known) \quad \dag \in \{R,U\}\\
(\textit{Conf}) & [S]\diam{U}\diam{R}\diam{S}\diam{S}(\known \land
\lnot \diam{R}\known \land \diam{U}\known \land
\diam{R}\diam{U}(\known \land \lnot \diam{R}\known))\\
(\textit{UR-Irr}) & [S][U][R]\lnot \known \land [S][R][U]\lnot \known\\
(\textit{URU-noCycle}) & [S][U][R][U]\lnot \known\\
(\textit{Unique}) & [S] \left( \bigvee_{1\leq i\leq n} t_i \wedge \bigwedge_{1\leq i < j \leq n } (t_i\to \lnot t_j)\right) \\
(\textit{Vert}) & [S] \bigwedge_{1\leq i\leq n} \left( t_i \to \diam{U} \bigvee_{1\leq j\leq n,u(T_i)=d(T_j)}  t_j\right) \\
(\textit{Horiz}) & [S] \bigwedge_{1\leq i\leq n} \left( t_i \to
\diam{r} \bigvee_{1\leq j\leq n,r(T_i)=l(T_j)}  t_j\right)
\end{array}
\end{displaymath}
Let the formula $\varphi^T$ be the conjunction of all the above
formulas. One can show that $T$ tiles $\omega\times\omega$ iff
$\varphi^T$ is satisfiable, following the ideas used in the proof of
Theorem~\ref{thm:tle_undecidable}.

%Suppose $\model,w\models\varphi^T$. Observe that (\textit{Back}) and
%(\textit{Spy}) impose $w$ to be a spy point over all its
%$S$-accessible states of $\model$ (and also force $U$ and $R$ to be
%irreflexive and asymmetric). These $S$-accessible states will be the
%tiles. From this it follows that $[S]\psi$ holds at $w$ iff $\psi$
%is true at every tile. Additionally, $\diam{S}\diam{S}\psi$ holds at
%tile $v$ iff $\psi$ is true at some tile (maybe the same one).
%
%Taking the above points into account, one can establish the
%following. (\textit{Grid}) states that from every tile there is
%another tile moving up (that is, following the $U$-relation). The
%same holds for the right direction (following the $R$-relation).
%(\textit{Func}) forces that $U$ and $R$ are both functionals, given
%that (\textit{Back}) and (\textit{Spy}) guarantee irreflexivity and
%asymmetry of $U$ and $R$ respectively. (\textit{Conf}) imposes that
%the tiles are arranged in a grid pattern. To make its job,
%(\textit{Conf}) needs the composed relation $U\circ R$ to be
%irreflexive, and to avoid a cycle with the $URU$ pattern. This job
%is done by (\textit{UR-Irr}) and (\textit{URU-noCycle})
%respectively.
%
%All the formulas we discuss up to now configure the grid. The last
%three ensure that every tile has a unique type $t_i$, and that the
%colors of the tiles match properly. From this, it easily follows
%that $\model$ is a tiling of $\omega\times\omega$.
%
%For the converse, suppose $f:\omega\times\omega\to T$ is a tiling of
%$\omega\times\omega$. We define the model
%$\model=\diam{W,\{S,U,R\},V,\emptyset}$ as follows:
%\begin{itemize}
%\item $W=\omega\times\omega \cup \{w\}$
%\item $S=\{(w,v),(v,w)\mid v\in\omega\times\omega\}$  (hence $w$ will act as the spy
%point)
%\item $U=\{((x,y),(x,y+1))\mid x,y\in\omega\}$
%\item $R=\{((x,y),(x+1,y))\mid x,y\in\omega\}$
%\item $V(p)=\{w\}$; $V(t_i)=\{x\mid x\in\omega\times\omega, f(x)=T_i\}$
%\end{itemize}
%The reader may verify that, by construction,
%$\model,w\models\varphi^T$.
%
\end{pf}

We close this section showing that only one nominal is enough to
turn $\tlm$ undecidable. We call this language $\tlmi$.

\begin{thm}
There is a formula $\textit{Inf} \in \tlmi$ such that $\mathcal{M},w
\models \textit{Inf}$ implies that the domain of $\mathcal{M}$ is an
infinite set.
\end{thm}

\begin{pf}
Consider the following formulas:
$$
\begin{array}{rl}
(\textit{Back}) & i \land [R] \lnot i  \land \diam{R} \top \land [R]\diam{R} i \\
(\textit{Empty}) & [R]\lnot \known \land [R][R](\lnot i \to \lnot \known)\\
(\textit{Spy}) & [R][R]( \lnot i \to \diam{R} (i \land \diam{R} ( \known \land \lnot \diam{R} ( \known \land \lnot i))))\\
%(\textit{Irr}) & \lnot \diam{R} \diam{R} ( \lnot i \land \known))\\
(\textit{Succ}) & [R]\diam{R} \lnot i \\
(\textit{no-3cyc}) & \lnot \diam{R} \diam{R} ( \lnot \known \land \diam{R} ( \lnot \known \land \diam{R} ( \known \land \lnot i)))\\
(\textit{Tran}) & [R] \diam{R} ( i \land [R] ( \lnot \known \to \diam{R}( i \land \diam{R} ( \known \land  \diam{R}(\known \land \lnot i) ))))\\
\end{array}
$$

Let $\textit{Inf}$ be $\textit{Back} \land \textit{Spy} \land
\textit{Irr} \land \textit{Succ} \land \textit{3cyc} \land
\textit{Tran}$. Let $\model=\diam{W, R, V, \emptyset}$. One can
follow the idea used in the proof of
Theorem~\ref{thm:infinite_model} to show that if $\model, w \models
\textit{Inf}$, then $W$ is infinite.
%
%Suppose $\model, w \models \textit{Inf}$. Let $B = \{b \in W \mid
%wRb\}$. Because $\textit{Back}$ is satisfied, $w \not \in B$, $B
%\not= \emptyset$ and for all $b \in B$, $bRw$. Note that
%\textit{Empty} sais that the one and two-step horizon is not known,
%and also this implies that every state in $B$ is irreflexive.
%Because $\textit{Spy}$ is satisfied, if $a \not= w$ and $a$ is a
%successor of an element of $B$ then $a$ is also an element of $B$.
%As $\textit{Succ}$ is satisfied at $w$, every point in $B$ has a
%successor distinct from $w$. As $\textit{3cyc}$ is satisfied, there
%cannot be $3$ different elements in $B$ forming a cycle, and this
%sentence together with $\textit{Tran}$ force $R$ to transitively
%order $B$.
%
%It follows that $B$ is an unbounded strict partial order, hence
%infinite, and so is $W$.
\end{pf}


We show that this logic is undecidable.

\begin{thm}
The satisfiability problem for $\tlmi$ is undecidable.
\end{thm}
\begin{pf}
Let $T=\{T_1,\dots,T_n\}$ be a set of tile types. Given a tile type
$T_i$, $u(T_i)$, $r(T_i)$, $d(T_i)$, $l(T_i)$ will represent the
colors of the up, right, down and left edges of $T_i$ respectively.
Define:

\begin{displaymath}
\begin{array}{rl}
(\textit{Back}) & i \land [S]\lnot i \land \diam{S} \top \land [S]\diam{S}i \land [S][S]i\\
(\textit{Empty}) & [S]\lnot \known \land [S][S](\lnot i \to \lnot \known) \land [S][\dag]\lnot \known \quad \dag \in \{R,U\}\\
(\textit{Spy}) & [S][\dag]\diam{S}(i \land \diam{S}(\known \land \lnot\diam{\dag}\known)) \quad \dag \in \{R,U\}\\
(\textit{Grid}) & [S]\diam{\dag} \top \quad \dag \in \{R,U\}\\
(\textit{Func}) & [S][\dag]\diam{S}\diam{S}(\known \land \diam{\dag}\known \land [\dag]\known) \quad \dag \in \{R,U\}\\
(\textit{Conf}) & [S]\diam{U}\diam{R}\diam{S}\diam{S}(\known \land \lnot \diam{R}\known \land \diam{U}\known \land \diam{R}( \lnot \known \land (\diam{U}(\known \land \lnot \diam{R}\known)))\\
(\textit{UR-noCycle}) & [S][U][R]\lnot \known \land [S][R][U]\lnot \known\\
(\textit{URU-noCycle}) & [S][U][R][U]\lnot \known\\
(\textit{Unique}) & [S] \left( \bigvee_{1\leq i\leq n} t_i \wedge \bigwedge_{1\leq i < j \leq n } (t_i\to \lnot t_j)\right) \\
(\textit{Vert}) & [S] \bigwedge_{1\leq i\leq n} \left( t_i \to \diam{U} \bigvee_{1\leq j\leq n,u(T_i)=d(T_j)}  t_j\right) \\
(\textit{Horiz}) & [S] \bigwedge_{1\leq i\leq n} \left( t_i \to \diam{r} \bigvee_{1\leq j\leq n,r(T_i)=l(T_j)}  t_j\right) \\
\end{array}
\end{displaymath}
Let the formula $\varphi^T$ be the conjunction of all the above
formulas. One can verify that $T$ tiles $\omega\times\omega$ iff
$\varphi^T$ is satisfiable, following the ideas used in the proof of
Theorem~\ref{thm:tle_undecidable}.
%
%Suppose $\model,w\models\varphi^T$. Observe that (\textit{Back}) and
%(\textit{Spy}), together with (\textit{Empty}) impose $w$ to be a
%spy point over all its $S$-accessible states of $\model$ (and also
%force $U$ and $R$ to be irreflexive and asymmetric). These
%$S$-accessible states will be the tiles. From this it follows that
%$[S]\psi$ holds at $w$ iff $\psi$ is true at every tile.
%Additionally, $\diam{S}\diam{S}\psi$ holds at tile $v$ iff $\psi$ is
%true at some tile (maybe the same one).
%
%Taking the above points into account, one can establish the
%following. (\textit{Grid}) states that from every tile there is
%another tile moving up (that is, following the $U$-relation). The
%same holds for the right direction (following the $R$-relation).
%(\textit{Func}) forces that $U$ and $R$ are both functionals, given
%that (\textit{Back}) and (\textit{Spy}) guarantee irreflexivity and
%asymmetry of $U$ and $R$ respectively. (\textit{Conf}) imposes that
%the tiles are arranged in a grid pattern. To make its job,
%(\textit{Conf}) needs the composed relation $U\circ R$ to be
%irreflexive, and to avoid a cycle with the $URU$ pattern. This job
%is done by (\textit{UR-Irr}) and (\textit{URU-noCycle})
%respectively.
%
%All the formulas we discuss up to now configure the grid. The last
%three ensure that every tile has a unique type $t_i$, and that the
%colors of the tiles match properly. From this, it easily follows
%that $\model$ is a tiling of $\omega\times\omega$.
%
%For the converse, suppose $f:\omega\times\omega\to T$ is a tiling of
%$\omega\times\omega$. We define the model
%$\model=\diam{W,\{S,U,R\},V,\emptyset}$ as follows:
%\begin{itemize}
%\item $W=\omega\times\omega \cup \{w\}$
%\item $S=\{(w,v),(v,w)\mid v\in\omega\times\omega\}$  (hence $w$ will act as the spy
%point)
%\item $U=\{((x,y),(x,y+1))\mid x,y\in\omega\}$
%\item $R=\{((x,y),(x+1,y))\mid x,y\in\omega\}$
%\item $V(p)=\{w\}$; $V(t_i)=\{x\mid x\in\omega\times\omega, f(x)=T_i\}$
%\end{itemize}
%The reader may verify that, by construction,
%$\model,w\models\varphi^T$.
%
\end{pf}
