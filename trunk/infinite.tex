The last issue that we will discuss in this paper
is the undecidability of the satisfiability problem
for both $\tl$ and $\tle$.  The proof is an adaptation
of the proof of undecidability of $\hlogic$ presented in~\cite{BS95}.

We first prove that both languages lack the finite model property~\cite{BRV01}.

\begin{thm}\label{thm:infinite_model}
There is a formula {\em $\textit{Inf} \in \tle$} such that $\mathcal{M},w \models \textit{Inf}$ implies that the domain of $\mathcal{M}$ is an infinite set.
\end{thm}

\begin{pf}
Consider the following formulas:
$$
\begin{array}{rl}
(\textit{Back}) & p \land [r] \lnot p  \land \diam{r} \top \land \remember ([r]\diam{r} \known) \\
(\textit{Spy}) & \remember( [r][r]( \lnot p \to \remember ( \diam{r} (p \land \known \land \diam{r}(\lnot p \land \known))))) \\
(\textit{Irr}) & [r] \remember \lnot \diam{r} \known\\
(\textit{Succ}) & [r]\diam{r} \lnot p \\
(\textit{3cyc}) & \lnot(\diam{r}\remember\diam{r}(\lnot p \land \diam{r}(\lnot p \land \lnot \known \land \diam{r}\known)))\\
(\textit{Tran}) & [r] \remember [r](\lnot p \to ([r] (\lnot p \to (\remember \diam{r}(p \land \diam{r}(\known \land \diam{r}\known)))))) \\
\end{array}
$$

Let $\textit{Inf}$ be $\textit{Back} \land \textit{Spy} \land
\textit{Irr} \land \textit{Succ} \land \textit{3cyc} \land
\textit{Tran}$. Let $\model=\diam{W, R, V, \emptyset}$. We show that
if $\model, w \models \textit{Inf}$, then $W$ is infinite.

Suppose $\model, w \models \textit{Inf}$. Notice that if $\known$
holds in a state, is because it was previously remembered by the
evaluating formula. Let $B = \{b \in W \mid wRb\}$. Because
$\textit{Back}$ is satisfied, $w \not \in B$, $B \not= \emptyset$
and for all $b \in B$, $bRw$. Because $\textit{Spy}$ is satisfied,
if $a \not= w$ and $a$ is a successor of an element of $B$ then $a$
is also an element of $B$. As $\textit{Irr}$ is satisfied at $w$,
every state in $B$ is irreflexive. As $\textit{Succ}$ is satisfied
at $w$, every point in $B$ has a successor distinct from $w$. As
$\textit{3cyc}$ is satisfied, there cannot be $3$ different elements
in $B$ forming a cycle, and this sentence together with
$\textit{Tran}$ force $R$ to transitively order $B$.

It follows that $B$ is an unbounded strict partial order, hence
infinite, and so is $W$.
\end{pf}

To prove failure of the finite model property for the case $\tl$
we first notice that the following lemma is easy to establish (we only
state it for the monomodal case; a similar result is true in the
multimodal case).
Failure of the finite model property is then a direct consequence.

\begin{lem}\label{lem:boxes}
Let $\varphi$ be a formula of modal depth $d$. If
{\em $\diam{W,R_r,V,S},w\models$} \linebreak {\em $\left(\bigwedge_{i=0}^d [r]^i \lnot
\known\right) \wedge \varphi$} then $\diam{W,R_r,V,\emptyset},w\models
\varphi$.
\end{lem}

\begin{cor}
{\em $\tl$} lacks the finite model property.
\end{cor}

\begin{pf}
Using Lemma~\ref{lem:boxes}, one can easily see that the formula
$\textit{Inf}\land \left(\bigwedge_{i=0}^4 [r]^i \lnot
\known\right)$, where $\textit{Inf}$ is the one in the proof of
Theorem~\ref{thm:infinite_model}, forces an infinite model.
\end{pf}

We now turn to undecidability.
We show that $\tl$ and $\tle$ are undecidable by encoding the
$\omega\times\omega$ tiling problem (see~\cite{BGG97}). Following the idea
in~\cite{BS95}, we construct a spy point over the relation $S$ which
has access to every tile. The relations $U$ and $R$ represent moving
up and to the right, respectively, from one tile to the other. We code each
type of tile with a fixed propositional symbol $t_i$. With this
encoding we define for each tiling problem $T$, a formula $\varphi^T$ such that the set of tiles $T$ tiles
$\omega\times\omega$ iff $\varphi^T$ has a model.

\begin{thm} \label{thm:tle_undecidable}
The satisfiability problem for {\em $\tle$} is undecidable.
\end{thm}

\begin{pf}
Let $T=\{T_1,\dots,T_n\}$ be a set of tile types. Given a tile type
$T_i$, $u(T_i)$, $r(T_i)$, $d(T_i)$, $l(T_i)$ will represent the
colors of the up, right, down and left edges of $T_i$ r espectively.
Define
$$
\begin{array}{rl}
(\textit{Back}) & p \land [S] \lnot p  \land \diam{S} \top \land \remember ([S]\diam{S} \known) \land \remember ([S][S] \known) \\
(\textit{Spy}) & \remember [S][\dag]\remember\diam{S}(\known\wedge p \wedge \diam{S}(\known \wedge \lnot p)) ,\qquad\textrm{where } \dag\in\{U,R\}\\
(\textit{Grid}) & [S][U]\lnot p \wedge [S][R]\lnot p \wedge [S]\diam{U}\top \wedge [S]\diam{r}\top \\
(\textit{Func}) & \remember[S]\remember\diam{\dag}\remember\diam{S}\diam{S}(\known\wedge\diam{\dag}\known \wedge [\dag]\known),\qquad\textrm{where } \dag\in\{U,R\}\\
(\textit{Irr}) & [S] \remember [\dag] \lnot \known, \qquad\textrm{where } \dag\in\{U,R\}\\
(\textit{2cyc}) & [S] \remember [\dag] [\dag] \lnot \known, \qquad\textrm{where } \dag\in\{U,R\}\\
(\textit{Confluent}) & [S] \remember \diam{U}\diam{r} \remember \diam{S}\diam{S} (\known \wedge \diam{U}\diam{r}\known \wedge \diam{r}\diam{U}\known)\\
(\textit{UR-Irr}) & [S] \remember [U][R] \lnot \known\\
(\textit{UR-2cyc}) & [S] \remember [U] [R] [U] [R]\lnot \known \\
(\textit{Unique}) & [S] \left( \bigvee_{1\leq i\leq n} t_i \wedge \bigwedge_{1\leq i < j \leq n } (t_i\to \lnot t_j)\right) \\
(\textit{Vert}) & [S] \bigwedge_{1\leq i\leq n} \left( t_i \to \diam{U} \bigvee_{1\leq j\leq n,u(T_i)=d(T_j)}  t_j\right) \\
(\textit{Horiz}) & [S] \bigwedge_{1\leq i\leq n} \left( t_i \to \diam{r} \bigvee_{1\leq j\leq n,r(T_i)=l(T_j)}  t_j\right) \\
\end{array}
$$

Let the formula $\varphi^T$ be the conjunction of all the above
formulas. We show that $T$ tiles $\omega\times\omega$ iff
$\varphi^T$ is satisfiable.

Suppose $\model,w\models\varphi^T$. Observe that (\textit{Back}) and
(\textit{Spy}) impose $w$ to be a spy point over all its
$S$-accessible states of $\model$. These $S$-accessible states will
be the tiles. From this it follows that $[S]\psi$ holds at $w$ iff
$\psi$ is true at every tile. Additionally, $\diam{S}\diam{S}\psi$
holds at tile $v$ iff $\psi$ is true at some tile (maybe the same
one).

Taking the above points into account, one can establish the
following. (\textit{Grid}) states that from every tile there is
another tile moving up (that is, following the $U$-relation). The
same holds for the right direction (following the $R$-relation).
(\textit{Func}) forces that $U$ and $R$ are both functionals, given
that (\textit{Irr}) and (\textit{2cyc}) guarantee irreflexivity and
asymmetry of $U$ and $R$ respectively. (\textit{Confluent}) imposes
that the tiles are arranged in a grid pattern. To make its job,
(\textit{Confluent}) needs the composed relation $U\circ R$ to be
irreflexive and asymmetric, and this is done by (\textit{UR-Irr})
and (\textit{UR-2cyc}) respectively.

All the formulas we discuss up to now configure the grid. The last
three ensure that every tile has a unique type $t_i$, and that the
colors of the tiles match properly. From this, it easily follows
that $\model$ is a tiling of $\omega\times\omega$.

For the converse, suppose $f:\omega\times\omega\to T$ is a tiling of
$\omega\times\omega$. We define the model
$\model=\diam{W,\{S,U,R\},V,\emptyset}$ as follows:
\begin{itemize}
\item $W=\omega\times\omega \cup \{w\}$
\item $S=\{(w,v),(v,w)\mid v\in\omega\times\omega\}$  (hence $w$ will act as the spy
point)
\item $U=\{((x,y),(x,y+1))\mid x,y\in\omega\}$
\item $R=\{((x,y),(x+1,y))\mid x,y\in\omega\}$
\item $V(p)=\{w\}$; $V(t_i)=\{x\mid x\in\omega\times\omega, f(x)=T_i\}$
\end{itemize}
The reader may verify that, by construction,
$\model,w\models\varphi^T$.
\end{pf}


\begin{cor}
The satisfiability problem for {\em $\tl$} is undecidable.
\end{cor}
%
\begin{pf}
Using Lemma~\ref{lem:boxes} and the formula $\varphi^T$ in Theorem~\ref{thm:tle_undecidable}, we obtain a formula $\psi$ such that if $\model,w\models\psi$ then $\model$ is a tiling of $\omega \times \omega$. For the converse, we can build exactly the same model as in the above proof.
\end{pf}
