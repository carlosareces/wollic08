\section{$\MS$}



% We will denote by $\MSpp$ to the fragment of $\MS$ that does not
% make use of $\pop$, only $\push$ and $\tope$ are allowed.
%
% The fragment $\MSbound{k}$ restricts the logic to a bounded stack
% not bigger than $k$. Any attempt to push more than $k$ elements will
% lead to a false valuation. Hence the rule of $\push$ is now:
% \begin{center}
% \begin{tabular}{rcl}
% $\diam{W,\rels,V,S},w \models \push \varphi$ &
%  iff & $|S|<k$ and $\diam{W,\rels,V,S\cdot w},w \models \varphi$
% \end{tabular}
% \end{center}
%
% \begin{pro}
% $\hlogicone$, $\MSpp$ and $\MSbound{1}$ have the same expressive
% power.
% \end{pro}
% \begin{pf}
% The fact that $\hlogicone$ is not more expressive than $\MSpp$ can
% be shown using the same translation as in
% Proposition~\ref{prop:hl_leq_stack}, we just have to note that no
% $\pop$ will result from the translation of $\hlogicone$ formulas.
%
% For the converse we can use almost the same translation as in
% Proposition~\ref{prop:stack_leq_hl} with a slight change in the
% rule:
% \begin{eqnarray*}
% \Tr_N(\push \varphi) & = & \down x. \Tr_{N\cdot x}(\varphi)
% \end{eqnarray*}
% Where $x$ is the only variable of $\hlogicone$.
%
% \smallskip
%
% Equivalently, to show that $\hlogicone$ is not more expressive than
% $\MSbound{1}$:
% \begin{eqnarray*}
% \Tr_N(\down x.\varphi) & = & \pop\push\Tr_{x}(\varphi) \quad \textrm{if } |N|=1\\
% \Tr_N(\down x.\varphi) & = & \push\Tr_{x}(\varphi) \quad \textrm{otherwise}\\
% \Tr_N(x) & = & \tope
% \end{eqnarray*}
% And for the converse, we can use the same translation as in
% Proposition~\ref{prop:stack_leq_hl}. We just have to realize that
% $N$ will have at most length 1 at all times, making possible the use
% of just one variable $x$.
% \end{pf}
