\section{$\MS$}

\begin{pro}\label{prop:stack_leq_hl}
$\MS$ is not more expressive than $\hlogic$.
\end{pro}

\begin{pf}
We show that every property captured by a $\MS$-formula can be
captured by a $\hlogic$-formula. We do this by defining the
following translation transforming an $\MS$-formula and a list of
nominals $N$ into an $\hlogic$-formula.
\begin{eqnarray*}
\Tr_N(p) & = & p \quad p \in \prop\\
\Tr_N(\lnot \varphi) & = & \lnot \Tr_N(\varphi) \\
\Tr_N(\diam{r} \varphi) & = & \diam{r} \Tr_N(\varphi) \\
\Tr_N(\varphi_1 \land \varphi_2) & = & \Tr_N(\varphi_1) \land \Tr_N(\varphi_2)\\
\Tr_N(\push \varphi) & = & \down i. \Tr_{N\cdot i}(\varphi) \quad
\textrm{where $i \notin N$.}\\
\Tr_{N\cdot i}(\pop \varphi) & = & \Tr_{N}(\varphi)\\
\Tr_{[\ ]}(\pop \varphi) & = & \lnot\top\\
\Tr_{N\cdot i}(\tope) & = & i\\
\Tr_{[\ ]}(\tope) & = & \lnot\top
\end{eqnarray*}
(Here $q$ is a fixed propositional symbol in \prop.)

It can be shown by induction in $\varphi$ that $\model, w \models
\varphi$ iff $\model, g, w \models \Tr_{[\ ]}(\varphi)$, for any $g$
\end{pf}

\begin{pro} \label{prop:hl_leq_stack}
$\hlogic$ is not more expressive than $\MS$.
\end{pro}

\begin{pf}
As in the previous proof, we define a translation transforming an
$\hlogic$-formula and a list of nominals $N$ into an $\MS$-formula.
The idea is that in a formula of the form $\down i.\down j.\varphi$
we can stack first $i$ and then $j$. Now if $i$ and $j$ appear in
$\varphi$, then $j$ must be translated as $\tope$ and  $i$ must be
translated as $\pop\tope$. We define the translation which coincides
with the one defined in the proof of
Proposition~\ref{prop:stack_leq_hl} for the propositional, negation,
conjunction and modality, plus:
\begin{eqnarray*}
\Tr_N(\down i.\varphi) & = & \push\Tr_{N\cdot i}(\varphi)\\
\Tr_N(i) & = & \pop^{\len{N}-x}\tope \quad i \in \nom, N[x]=i,
\forall y>x:\, N[y]\not=i
\end{eqnarray*}
(Here $\len{N}$ represents the length of $N$; for convenience the
lists are numbered from $1$ and $N[x]$ represents the $x$-th element
of $N$.)

It can be shown by induction in $\varphi$ that if $\varphi$ is an
$\hlogic$-formula with no free nominals, $\model, g, w \models
\varphi$ iff $\model, w \models \Tr_{[\ ]}(\varphi)$ for any $g$.
\end{pf}

\begin{cor}
$\MS$ and $\hlogic$ have the same expressive power.
\end{cor}

We will denote by $\MSpp$ to the fragment of $\MS$ that does not
make use of $\pop$, only $\push$ and $\tope$ are allowed.

The fragment $\MSbound{k}$ restricts the logic to a bounded stack
not bigger than $k$. Any attempt to push more than $k$ elements will
lead to a false valuation. Hence the rule of $\push$ is now:
\begin{center}
\begin{tabular}{rcl}
$\diam{W,\rels,V,S},w \models \push \varphi$ &
 iff & $|S|<k$ and $\diam{W,\rels,V,S\cdot w},w \models \varphi$
\end{tabular}
\end{center}

\begin{pro}
$\hlogicone$, $\MSpp$ and $\MSbound{1}$ have the same expressive
power.
\end{pro}
\begin{pf}
The fact that $\hlogicone$ is not more expressive than $\MSpp$ can
be shown using the same translation as in
Proposition~\ref{prop:hl_leq_stack}, we just have to note that no
$\pop$ will result from the translation of $\hlogicone$ formulas.

For the converse we can use almost the same translation as in
Proposition~\ref{prop:stack_leq_hl} with a slight change in the
rule:
\begin{eqnarray*}
\Tr_N(\push \varphi) & = & \down x. \Tr_{N\cdot x}(\varphi)
\end{eqnarray*}
Where $x$ is the only variable of $\hlogicone$.

\smallskip

Equivalently, to show that $\hlogicone$ is not more expressive than
$\MSbound{1}$:
\begin{eqnarray*}
\Tr_N(\down x.\varphi) & = & \pop\push\Tr_{x}(\varphi) \quad \textrm{if } |N|=1\\
\Tr_N(\down x.\varphi) & = & \push\Tr_{x}(\varphi) \quad \textrm{otherwise}\\
\Tr_N(x) & = & \tope
\end{eqnarray*}
And for the converse, we can use the same translation as in
Proposition~\ref{prop:stack_leq_hl}. We just have to realize that
$N$ will have at most length 1 at all times, making possible the use
of just one variable $x$.
\end{pf}
