\section{Conclusions and Further Work}

In this article we investigate several memory logics.  These logics were inspired by
the hybrid logic $\hlogic$: the $\down$ operator can be thought of
as a storage command, and our aim is to carry this idea further,
investigating different ways in which information can be stored and retrieved. We
have proved that, in terms of  expressive power, the memory logics
we presented lay between the basic modal logic $\K$ and the hybrid
logic $\hlogic$.  In most cases ($\cMLRK$, $\cMLRKE$, $\cMLRKME$ and
$\cMLS$) the language obtained were still very expressive and their
satisfiability problem undecidable.  $\cMLRKM$, on the other hand,
has the bounded tree model property, and its
satisfiability problem is \textsc{pspace}-complete.

When we are interested in using modal logics as tools for
modeling behaviour, it is natural to look for extensions
which are able to capture some notion of state.
 Good examples of such logics are the different epistemic logics with dynamic
operators (often called Dynamic Epistemic Logics~\cite{epistemic}),
which model the evolution of knowledge by accessing and
changing the model structure through logic
operators~\cite{plaza,1028135,1225972}. Many other
examples exist in the literature: update
logics~\cite{vanbenthem05,gerbrandy99}, XCTL~\cite{113765}, the
freeze operator~\cite{Alur89areally,Henzinger90half-ordermodal},
etc.

The family of memory logics presented in
this article seems to capture some common features which are shared by
logics like the ones we just mentioned.  In this article, we propose
to take them as a starting point, aiming to define a general framework
where we can study how to add explicit state to a model, and how to
access and modify it via logical operators.

We have investigated only a small number of memory logics.  We can
define other natural operators that fit nicely within this perspective.
Two particularly
interesting cases are, for example, the ``forget'' operator $\forget$, which
eliminates the current point of evaluation from the memory $S$, and the
``erase'' operator $\erase$, that completely wipes out the memory:

\begin{center}
\begin{tabular}{rcl}
$(\gM,S),w \models \forget \varphi$ &
 iff & $(\gM,S\setminus\cset{w}),w \models \varphi$ \\
$(\gM,S),w \models \erase \varphi$ &
 iff & $(\gM,\emptyset),w \models \varphi$.
\end{tabular}
\end{center}
%
We have proved that the addition of $\forget$ and $\erase$ to
$\cMLRK$ actually adds expressive power, but we have not completely
determined the expressivity relationship between these new operators
and $\HL(\down)$. We are also working on complete axiomatizations,
and we have found complete systems for $\cMLRKM$; and also for $\cMLRK$,
$\cMLRKE$ and $\cMLRKME$ when nominals are added to the
language~\cite{comp}. Furthermore, once nominals have been added,
the internalized hybrid tableau method~\cite{backburn00:_inter} can
be naturally adapted, leading to complete tableau algorithms.

% \tb{SS: Mencionamos algo mas en trabajo futuro relacionado con memory logics sobre estructuras lineales? Tambien se podria mencionar el estudio sobre interpolacion y Beth}

% characterizations. Extending the language with nominals is a natural
% step, and then adapting the internalized hybrid tableau
% method~\cite{backburn00:_inter} to the new languages is
% straightforward.  More interesting is to explore new languages of
% the family (like $\texttt{(push)}$, $\texttt{(pop)}$, or
% $\texttt{(forget)}$), and interaction between the memory operators
% and the modalities.
%$\diam{r}$ and $[r]$ modalities.


% \tb{extend relation with DEL, change ending, weak}
% The work presented in this paper is somehow related in spirit with
% the work on Dynamic Epistemic Logic and other update
% logics~\cite{vanbenthem05,gerbrandy99}, but as we discuss in the
% introduction, our inspiration was rooted in a new interpretation of
% the $\down$ binder.
