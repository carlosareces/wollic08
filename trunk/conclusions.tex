\section{Conclusions and Further Work}

In this paper we investigate several members of a family of logics
that we called \emph{memory logics}.  These logics were inspired by
the hybrid logic $\hlogic$: the $\down$ operator can be thought of
as a storage command, and our aim is to carry this idea further
investigating different ways in which information can be stored. We
have proved that, in terms of  expressive power, the memory logics
we presented lay between the basic modal logic $\K$ and the hybrid
logic $\hlogic$.  In some cases, the reduced expressive power is not
sufficient to ensure good computational behavior: both $\cMLRK$,
$\cMLRKE$ and $\cMLRKME$ fail to have the finite model property, and
moreover, their satisfiability problems are undecidable. On the
other hand, $\cMLRKM$ has the bounded tree model property, and its
satisfiability problem is \textsc{pspace}-complete.

The efforts to increase modal logic expressivity by adding some
notion of state are the result of a natural need, since modal logics
are used in many different scenarios as tools for modeling behavior.
In this sense, there is work already done that tries to capture this
feature. A good example are the epistemic logics with dynamic
operators (often called Dynamic Epistemic Logics~\cite{epistemic}),
which allow to express the evolution of knowledge by accessing and
changing the model structure through logic
operators~\cite{plaza,1028135,1225972}. There are many other
examples in this direction: update
logics~\cite{vanbenthem05,gerbrandy99}, XCTL~\cite{113765}, the
freeze operator~\cite{Alur89areally,Henzinger90half-ordermodal},
etc. We believe that the memory logics family defines a common and
more general framework to add explicit state to a model, and to
access and modify it via logical operators.

Much work rest to be done. There are natural operators that can be
added when using a set as the information storage. Two particular
interesting cases are the ``forget'' operator $\forget$, which
eliminates the current point of evaluation from the set $S$, and the
``erase'' operator $\erase$, that completely wipes out the memory
$S$:

\begin{center}
\begin{tabular}{rcl}
$\diam{W,\rels,V,S},w \models \forget \varphi$ &
 iff & $\diam{W,\rels,V,S\setminus\cset{w}},w \models \varphi$ \\
$\diam{W,\rels,V,S},w \models \erase \varphi$ &
 iff & $\diam{W,\rels,V,\emptyset},w \models \varphi$.
\end{tabular}
\end{center}
%
We have proved that the addition of $\forget$ and $\erase$ to
$\cMLRK$ actually adds expressive power, but we have not completely
determined the expressivity relationship between these new operators
and $\HL(\down)$. We are also working on complete axiomatizations,
and we have found complete systems for $\cMLRKM$, and for $\cMLRK$,
$\cMLRKE$ and $\cMLRKME$ when nominals are added to the
language~\cite{comp}. Furthermore, once nominals have been added,
the internalized hybrid tableau method~\cite{backburn00:_inter} can
be naturally adapted, leading to complete tableau algorithms.

\tb{SS: Mencionamos algo mas en trabajo futuro relacionado con memory logics sobre estructuras lineales? Tambien se podria mencionar el estudio sobre interpolacion y Beth}

% characterizations. Extending the language with nominals is a natural
% step, and then adapting the internalized hybrid tableau
% method~\cite{backburn00:_inter} to the new languages is
% straightforward.  More interesting is to explore new languages of
% the family (like $\texttt{(push)}$, $\texttt{(pop)}$, or
% $\texttt{(forget)}$), and interaction between the memory operators
% and the modalities.
%$\diam{r}$ and $[r]$ modalities.


% \tb{extend relation with DEL, change ending, weak}
% The work presented in this paper is somehow related in spirit with
% the work on Dynamic Epistemic Logic and other update
% logics~\cite{vanbenthem05,gerbrandy99}, but as we discuss in the
% introduction, our inspiration was rooted in a new interpretation of
% the $\down$ binder.
