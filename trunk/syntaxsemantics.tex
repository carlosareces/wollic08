\section{Syntax and semantics for Memory Logics}

In this section we will introduce the syntax and semantics of the
different memory logics that we will discuss in the article, and
fix some terminology.

All the languages we will introduce are obtained by extending (in
one case, also slightly modifying) the syntax and semantics of the
basic modal logic. We are mainly interested in investigating the
behaviour of the $\remember$ and $\known$ operators introduced in the
previous section, but to put them in the appropriate context we will
actually investigate five different logics which are described in
Figure~\ref{logics}.

\newcommand{\cMLRK}{\ensuremath{\mathcal{L}_1}}
\newcommand{\cMLRKE}{\ensuremath{\mathcal{L}_2}}
\newcommand{\cMLRKM}{\ensuremath{\mathcal{L}_3}}
\newcommand{\cMLRKME}{\ensuremath{\mathcal{L}_4}}
\newcommand{\cMLS}{\ensuremath{\mathcal{L}_5}}
\newcommand{\ttup}[1]{\langle\!\langle #1 \rangle\!\rangle}
\newcommand{\bbox}[1]{[\![ #1 ]\!]}

\begin{figure}
\begin{center} \small
\begin{tabular}{|c|l|c|}\hline
Logic & Operators (in addition to the Boolean operators) & Class of Models \\ \hline
& & \\[-8pt]
\cMLRK & The basic modal operators $\tup{r}$ and $[r]$, & all\\
& plus the memory operators $\remember$ and $\known$ & \\ \hdashline[1pt/1pt]
& & \\[-8pt]
\cMLRKE & The basic modal operators $\tup{r}$ and $[r]$, & $S=\emptyset$\\
& plus the memory operators $\remember$ and $\known$ & \\ \hdashline[1pt/1pt]
& & \\[-8pt]
\cMLRKM & A variant of the modal operators  $\ttup{r}$ and $\bbox{r}$ & all\\
& plus the memory operators $\remember$ and $\known$ & \\ \hdashline[1pt/1pt]
& & \\[-8pt]
\cMLRKME & A variant of the modal operators  $\ttup{r}$ and $\bbox{r}$ & $S=\emptyset$\\
& plus the memory operators $\remember$ and $\known$ & \\ \hdashline[1pt/1pt]
& & \\[-8pt]
\cMLS & The basic modal operators $\tup{r}$ and $[r]$, & all\\
& plus the memory operators $\push$, $\pop$ and $\tope$ & \\ \hline
\end{tabular}
\end{center}
\caption{Memory Logics}~\label{logics}
\end{figure}


\begin{defn}[Syntax]\label{syntax}
Let $\prop=\{p_1, p_2, \dots\}$ (the \textit{propositional symbols})
and $\rel=\{r_1, r_2, \dots\}$ (the \textit{relational symbols}) be
pairwise disjoint, countable infinite sets of symbols. The set
$\forms$ of formulas  in the
signature $\langle \prop,\rel\rangle$ is defined as:
$$
\forms
::=      p
    \mid \lnot \varphi
    \mid \varphi_1 \land \varphi_2
    \mid \diam{r} \varphi
    \mid \ttup{r} \varphi
    \mid \known
    \mid \remember \varphi
    \mid \tope
    \mid \push \varphi
    \mid \pop \varphi,
$$

\noindent
where $p \in \prop$, $r \in \rel$  and $\varphi, \varphi_1,
\varphi_2 \in \forms$. The other standard operators are introduced
via definitions.  In particular $[r]\varphi := \neg\tup{r}\neg \varphi$
and $\bbox{r}\varphi := \neg\ttup{r}\neg \varphi$.
\end{defn}

Definition~\ref{syntax} introduces the syntax of the five logics
we will discuss in the paper. In particular, the operators allowed in
each language are the ones indicated in Figure~\ref{logics}.

\newcommand{\gL}{\mathcal{L}}

\begin{defn}[Semantics]\label{semantics}
Given a signature $\sig=\diam{\prop,\rel}$, a model is a tuple
$\gM = \diam{W, \rels, V, S}$ where
$W$ is a nonempty set, $\rels$ are binary relations over $W$, $V:\prop \to
\wp(W)$ is a valuation function. $S$ will be called the \emph{memory}
of the model, and for $\gL_i, i\in \cset{1,2,3,4}$ we will take $S$ as a set such that $S \subseteq W$,
while for $\cMLS$ we will take $S$ to be a list
of elements in $W$.

Given a model $\model = \diam{W, \rels, V, S}$
and list of states $[w_1,\dots,w_n]$, $w_i \in W$, we define $\model[w_1,\dots,w_n]$
as the model obtained by properly updating $S$ with $[w_1,\dots,w_n]$. For
$\gL_i, i\in \cset{1,2,3,4}$, $\model[w_1,\dots,w_n] =
\diam{W, \rels, V, S \cup \{ w_1,\dots,w_n\}}$.  For $\gL_5$, $\model[w_1,\dots,w_n] =
\diam{W, \rels, V, S \cdot [w_1,\dots,w_n]}$, where $\cdot$ is the concatenation operation
on lists.

For notational convenience, let us assume fixed for the rest of the article
the models $\gM = \diam{W, \rels, V, S}$, $\gM_1 = \diam{W_1, \relsu, V_1, S_1}$
and $\gM_2 = \diam{W_2, \relsd, V_2, S_2}$.

Let $\gM$ be a model, and $w \in W$ then
the semantics for the different operators is defined as:
$$
\begin{array}{rcl}
\gM, w \models p & \iff & w \in V(p) \\
\gM, w \models \neg \varphi & \iff & \gM, w \not\models \varphi\\
\gM, w \models \varphi \wedge \psi & \iff &
\gM, w \models \varphi \mbox{ and }
\gM, w \models \psi \\
\gM, w \models \diam{r}\varphi & \iff &
\mbox{there is $w'$ such that $R_r(w,w')$}
 \mbox{ and } \gM, w' \models \varphi\\
\gM, w \models \ttup{r}\varphi & \iff &
\mbox{there is $w'$ such that $R_r(w,w')$}
 \mbox{ and } \gM[w], w' \models \varphi\\
\gM, w \models \remember \varphi & \iff & \gM[w], w \models \varphi \label{remember}\\
\gM, w \models \known & \iff & w \in S \label{known}\\
\gM,w \models \push \varphi &
 \iff & \gM[w],w \models \varphi \\
\gM,w \models \texttt{(pop)} \varphi &
 \iff & \tup{W,\rels,V,S'},w \models \varphi\\
 & &  \mbox{ for some } S', w' \mbox{ such that } S = S'\cdot [w']\\
\gM,w \models \texttt{top} &
 \iff & S=S'\cdot[w] \mbox{ for some } S'\\
\end{array}
$$

The set of propositions that are true at a given state $w$ is defined as
$\props(w) = \{p \in \prop \mid w \in V(p)\}$. Given two models $\model_1$ and $\model_2$, and
states $w_1 \in W_1$ and $w_2 \in W_2$, we say that they \emph{agree} when $\props(w_1)=\props(w_2)$ and $w_1\in S_1$ iff $w_2
\in S_2$.

Given a model $\gM$ and $w$ in the domain of $\gM$, we call $\tup{\gM,w}$ a \emph{pointed
model}.
\end{defn}

%We say that $\varphi$ is \textit{valid} on a model $\model$ if for
%all states $w$, $\model, w \models \varphi$, and we write $\model
%\models \varphi$. We say that a formula is valid on a frame
%$\mathcal{F}= \diam{W, \rels}$ (written $\mathcal{F} \models
%\varphi$) iff for all valuations $V$ and all sets $S$, $\varphi$ is
%valid on $\diam{\mathcal{F}, V, S}$. Finally, we say that a formula
%$\varphi$ is valid (written $\models \varphi$) if $\mathcal{F}
%\models \varphi$ for all frames $\mathcal{F}$.

\newcommand{\gCE}{\mathcal{C}_\emptyset}

As we mentioned before, a particularly interesting class of models
to investigate is the class $\gCE = \{\gM \mid \gM =
\tup{W,\rels,V,\emptyset}\}$, i.e., the class of models where the
memory is empty.  It is over $\gCE$ that the operators $\known$ and
$\remember$ have the most natural interpretation, and as we will see
in the next section, the restriction to this class has important
effects on expressivity and decidability.


%
% We define also the notion of a formula being $\emptyset$-valid. A
% formula $\varphi$ is $\emptyset$-valid (written $\models_{\emptyset}
% \varphi$) iff $\model \models \varphi$ for all models
% $\model=\diam{M,\rels, V, S}$ such that $S = \emptyset$.
