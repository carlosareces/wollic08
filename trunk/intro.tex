\section{Memory Logics: Hybrid Logics with a Twist}\label{twist}

Nowadays, the term \emph{modal logics} is loosely used to cover
an extremely wide variety of languages, which are in turn used
in very different applications~\cite{BRV01,handbook}.  Actually,
we can say that the fact
that the number of members in this family keeps constantly increasing
is almost a defining characteristic.  While all modal logics have
certain general aspects in common (e.g., they are interpreted in
terms of relational structures, they are usually well behaved
computationally, etc.), one of the general traits of the field is
to investigate languages specially tailored for a particular task.

\tb{I don't like the actual intro.  I want to tone down the `memory
logics as a kind of hybrid logic' idea, and introduce more broadly
modal logics.}

Hybrid languages have been extensively investigated in the past
years~\cite{handbookarticle}. $\HL$, the simplest hybrid language, is usually presented as
the basic modal language $\K$ extended with special symbols (called
\emph{nominals}) to name individual states in a model. These new
symbols are simply a new sort of atomic symbols
$\cset{i,j,k,\ldots}$ disjoint from the set of standard
propositional variables. While they behave syntactically exactly as
propositional variables do, their semantic interpretation differ:
nominals denote elements in the model, instead of sets of elements.
This simple addition already results in increased expressive power.
For example the formula $i \wedge \diam{r}i$ is true in a state $w$,
only if $w$ is a reflexive point named by the nominal $i$. As the
basic modal language is invariant under unraveling, there is no
equivalent modal formula~\cite{BRV01}.

But as we said above, $\HL$ is just the simplest hybrid language.
Once nominals have been added to the language, other natural
extensions arise.  Having names for states at our disposal we can
introduce, for each nominal $i$, an operator $@_i$ that allows us to
jump to the point named by $i$ obtaining the language $\HL(@)$. The
formula $@_i\varphi$ (read `at $i$, $\varphi$') moves the point of
evaluation to the state named by $i$ and evaluates $\varphi$ there.
Intuitively, the $@_i$ operators internalize the satisfaction
relation `$\models$' into the logical language: $\model, w \models
\varphi \mbox{ ~iff~ } \model \models @_i\varphi$, where $i$ is a
nominal naming $w$. For this reason, these operators are usually
called \emph{satisfaction operators}.

If nominals are names for individual states, why not introduce also
binders?  We would then be able to write formulas like $\forall
i.\diam{r} i$, which will be true at a state $w$ if it is related to
all states in the domain.  The $\forall$ quantifier is very
expressive: the satisfiability problem of $\HL(\forall)$  ($\HL$
extended with the universal binder $\forall$) is
undecidable~\cite{BS95}. Moreover, $\HL(@,\forall)$ is expressively
equivalent to full first-order logic (over the appropriate
signature).

From a modal perspective, other binders besides $\forall$ are
possible.  The $\down$ binder binds nominals to the \emph{current}
point of evaluation. In essence, it enables us to create a name for
the here-and-now, and refer to it later in the formula.  For
example, the formula $\down i.\diam{r} i$ is true at a state $w$ if
and only if it is related to itself.  The intuitive reading is quite
straightforward: the formula says ``call the current state $i$ and
check that $i$ is reachable''.  The logic $\HL(\down)$ is also very
expressive but weaker than $\HL(\forall)$.  Sadly, its
satisfiability problem is also undecidable.

Different binders for hybrid logics have been investigated in detail
(see~\cite{BS95}), but in this article we want to take a look at
$\down$ from a slightly different perspective: we will consider
nominals and $\down$ as ways for storing and retrieving information
in the model.



\subsection{Models as Information Storage}

Nominals and $\downarrow$ work nicely together.
Whereas $\down i$ stores the current point of evaluation in the
nominal $i$, nominals act as checkpoints enabling us to retrieve
stored information by verifying if the current point is named by a
given nominal $i$.  To make this point clear, let's define formally
the semantics of $\HL(\down)$.

\begin{defn} A hybrid signature $\sig$ is a tuple $\diam{\prop,\rel,\nom}$
where $\prop$, $\rel$, $\nom$ are mutually disjoint infinite
enumerable sets (the sets of propositional symbols, relational
symbols and nominals, respectively).

Formulas of $\hlogic$ are defined over a given $\sig$ by the
following rules
$$
\forms ::= p \mid i \mid \lnot \varphi \mid \varphi_1 \land
\varphi_2 \mid \diam{r} \varphi \mid \down i. \varphi,
$$

\noindent
where $p \in \prop$, $i \in \nom$,  $r \in \rel$  and $\varphi,
\varphi_1, \varphi_2 \in \forms$. Formulas in which any nominal $i$
appears in the scope of a binder $\down i$ are called
\emph{sentences}.

A model for $\HL(\down)$ over a signature $\sig$ is a tuple
$\diam{W,\rels,V,g}$ where $\diam{W,\rels,V}$ is a standard Kripke
model (i.e., $W$ is a non empty set, each $R_r$ is a binary relation
over $W$, and $V$ is a valuation), and $g$ is an assignment function
from $\nom$ to $W$.

Given a model $\model = \diam{W,\rels,V,g}$ the semantic conditions
for the propositional and modal operators are defined as usual
(see~\cite{BRV01}), and in addition:
$$
\begin{array}{rcl}
\diam{W,\rels,V,g}, w \models  i & \mbox{iff} & g(i) = w\\
\diam{W,\rels,V,g}, w \models \down i.\varphi & \mbox{iff} &
  \diam{W,\rels,V,g^i_w}, w \models \varphi\\
& & \mbox{where $g^i_w$ is the assignment identical to $g$}\\
& & \mbox{except perhaps in that $g^i_w(i)=w$}.
\end{array}
$$
\end{defn}

We can think that $\down i$ is \emph{modifying} the model (by
storing the current point of evaluation into $i$), and that $i$ is
being evaluated in the modified model. We can see the assignment $g$
as a particular type of `information storage' in our model, and
consider $\down$ and $i$ as our way to access this information
storage for reading and writing.

\tb{rewrite paragraph below}
But let us take a step back and consider the new picture. When we
introduced the $\down$ binder, our main aim was to define a binder
which was weaker than the first-order quantifier.  We thought of the
semantics of $\down$ first, and we suitably adjusted the way we
updated the assignment later.   But why do we need to restrict
ourselves to binders and assignments?

Let us start with a standard Kripke models $\diam{W,\rels,V}$, and
let us consider a very simple addition: just a set $S \subseteq W$.
We can think of $S$ as a set of states that are, for
some reason, `known' to us. Already in this very simple set up we
can define the following operators
\begin{center}
\begin{tabular}{rcl}
$\diam{W,\rels,V,S},w \models \remember \varphi$ &
 iff & $\diam{W,\rels,V,S\cup\cset{w}},w \models \varphi$ \\
$\diam{W,\rels,V,S},w \models \known$ &
 iff & $w \in S$.
\end{tabular}
\end{center}

As it is clear from the semantic definition, the `remember' operator
$\remember$ (a unary modality) just marks the current state as being
`already visited', by storing it in our `memory' $S$. On the other
hand, the zero-ary operator $\known$ (for `known') queries $S$ to
check if the current state has already been visited.

In this simple language we would have that
$\diam{W,\rels,V,\emptyset},w \models \remember\diam{r} \known$ will
be true only if $R_r(w,w)$. Is this new logic equivalent to
$\HL(\down)$? As we will prove in this article, the answer is
negative: the new language is less expressive than $\HL(\down)$ but
 more expressive than $\K$. Intuitively, in the new language we
cannot discern between states stored in $S$, while an assignment $g$
keeps a complete mapping between states and nominals.

Naturally, we can include structures which are richer than a simple
set, in our models. For example,  $S$ could be an ordered list of
elements, and we could then consider operators that add or delete
particular elements in the list.

We will call this new family of logics \emph{Memory Logics}
and in this article we will focus on the $\remember$ and $\known$
operators. But, more generally, our proposal is to take seriously the usual saying
that `modal languages are languages to talk about labeled graphs'
but give us the freedom to choose what we want to `remember' about a
given graph and how we are going to store it.

