\section{Memory Logics: Modal Logics with a Twist}\label{twist}

Nowadays, the term \emph{modal logics} is loosely used to cover
an extremely wide variety of languages, which are in turn used
in very different applications~\cite{BRV01,blac:hand06}.  Actually,
we can say that the fact
that the number of members in this family keeps constantly increasing
is almost a defining characteristic.  While all modal logics have
certain general aspects in common (e.g., they are interpreted in
terms of relational structures, they are usually well behaved
computationally, etc.), one of the general traits of the field is
to investigate languages specially tailored for a particular task.

In this article we present a new family of modal logics that
we call \emph{memory logics}. They
are inspired by hybrid logics like $\HL(\down)$~\cite{arec:hybr05b}, but while the
$\down$ operator was introduced to investigate \emph{dynamic naming}
of elements in a model, we will be concerned with operators that
let us \emph{store} and \emph{retrieve} information from some kind
of \emph{information structure} or \emph{memory}.
We could actually consider $\HL(\down)$ as the first memory logic.
Let us be a bit more formal and introduce some basic concepts.
The syntax of $\hlogic$ is defined as follows
$$
\varphi ::= p \mid i \mid \lnot \varphi \mid \varphi \land
\varphi \mid \diam{r} \varphi \mid \down i. \varphi,
$$

\noindent
where $p$ is a propositional symbol, $i$ is a nominal (a new set of
atomic symbols) and  $r$ is a relational symbol (formulas in which any nominal $i$
appears in the scope of a binder $\down i$ are called \emph{sentences}).

We can see that the language of $\hlogic$ is the language of the basic
modal logic $\K$ (see \cite{BRV01} for details) extended with nominals
and the $\down$ binder.

Semantically, $\hlogic$ is also very close to $\K$.  $\hlogic$-formulas are also interpreted over Kripke models which are extended with an
assignment function to interpret nominals and $\down$. More formally,
a model for $\hlogic$ is a pair $(\gM,g)$ where $\gM=\diam{W,\rels,V}$ is a standard Kripke model (i.e., $W$ is a non empty set, each $R_r$ is a binary relation
over $W$, and $V$ is a valuation), and $g$ is an assignment function
such that for any nominal $i$, $g(i) \in W$.

Given then a Kripke model $\gM=\diam{W,\rels,V}$ and an assignment $g$,
the semantic conditions
for $\hlogic$ are defined as:
$$
\begin{array}{rcl}
(\gM,g), w \models  p & \mbox{iff} & w \in V(p)\\
(\gM,g), w \models  i & \mbox{iff} & g(i) = w\\
(\gM,g), w \models \neg \varphi & \mbox{iff} & (\gM,g), w \not \models \varphi \\
(\gM,g), w \models \varphi \wedge \psi & \mbox{iff} &
(\gM,g), w \models \varphi \mbox{ and } (\gM,g), w \models \psi\\
(\gM,g), w \models \tup{r} \varphi  & \mbox{iff} &
\mbox{there is $w'$ s.t.\ $R_r(w,w')$} \mbox{ and } (\gM,g), w' \models \varphi\\
(\gM,g), w \models \down i.\varphi & \mbox{iff} &
  (\gM,g'), w \models \varphi \mbox{ where $g'$ is identical to $g$}\\
& & \mbox{except perhaps in that $g'(i)=w$}.
\end{array}
$$

One way of looking to the semantic condition for $\down i.\varphi$ is
that it dynamically creates a name for the current state
(by linking the nominal $i$ to it), so that
we can later refer to it during the evaluation of $\varphi$.
An alternative perspective is to see $\down i$ as an
instruction to \emph{modify} the model (by
storing the current point of evaluation into $i$), and
continue the evaluation of $\varphi$ in the modified model.

The difference between the two perspectives is subtle, but
important for this article.  In the latter, we are considering
the assignment $g$ as a kind of \emph{memory}
in our model, while $\down i$ and $i$ are the tools we use to access
the memory for reading and writing.
The question then presents itself naturally: are there other
kinds of interesting memory structures and memory operators?

For example, we could say that the assignment $g$ is a very
sophisticated memory structure: it has unbounded size, it
provides direct access to all its memory cells, and each
stored element can be unequivocally retrieved.
Let us start with a standard Kripke model $\gM=\diam{W,\rels,V}$ and
consider a much simpler memory structure: just a set $S \subseteq W$.
We can think of $S$ as a set of states that are, for
some reason, `known' to us. Already in this very simple set-up we
can define the following operators:
\begin{center}
\begin{tabular}{rcl}
$(\gM,S),w \models \remember \varphi$ &
 iff & $(\gM,S\cup\cset{w}),w \models \varphi$ \\
$(\gM,S),w \models \known$ &
 iff & $w \in S$.
\end{tabular}
\end{center}

As it is clear from the semantic definition, the `remember' operator
$\remember$ (a unary modality) just marks the current state as being
`already visited', by storing it in our memory $S$. On the other
hand, the zero-ary operator $\known$ (for `known') queries $S$ to
check if the current state has already been visited.

In this simple language we would have, for example, that
$(\gM,\emptyset),w \models \remember\diam{r} \known$ will
be true only if $R_r(w,w)$. Is this new logic equivalent to
$\HL(\down)$? As we will prove in this article, the answer is
negative: the new language is less expressive than $\HL(\down)$ but
 more expressive than $\K$. Intuitively, in the new language we
cannot discern between states stored in $S$, while an assignment $g$
keeps a complete mapping between states and nominals.

Naturally, we can consider memory structures which are richer than a simple
set. For example,  $S$ could be an ordered list of
elements, and we could then consider operators that add or delete
particular elements in the list.

In this article we will focus on the $\remember$ and $\known$
operators defined above.  More generally, the main idea behind memory logics is to take seriously the usual saying
that `modal logics talk about labeled graphs'
but giving us the freedom to choose \emph{what} we want to remember about a
given graph, and \emph{how} we are going to store and retrieve it.

