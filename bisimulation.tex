\section{Model Equivalence}

In this section we will investigate the notion of model equivalence
for some of the memory logics that we introduced. Our goal is to
define tools that will help us investigate their expressive power.
In particular, we will define a notion of model equivalence in terms
of Ehrenfeucht-Fra\"iss\'e games~\cite{ebbi:math84} and then
introduce an alternative, but equivalent, notion in terms of
bisimulations.

We will focus on the languages including $\remember$, $\known$, $\tup{r}$ and
$\ttup{r}$ (i.e., the logics $\cMLRK$, $\cMLRKE$,
$\cMLRKM$ and $\cMLRKME$).  We will show in
Section~\ref{expressivity} that $\cMLS$ has the same expressive
power than $\HL(\down)$. Hence we don't need to introduce a new
notion of model equivalence for this logic, and we can use hybrid
bisimulations as introduced in~\cite{areces01:_hybrid}.

\newcommand{\EF}{\mathit{EF}}

\begin{defn}[Ehrenfeucht-Fra\"iss\'e Games]\label{def:EF-game}
Let
$\model_1$ and $\model_2$ be two models and let $w_1 \in W_1$
and $w_2 \in W_2$.

An \emph{Ehrenfeucht-Fra\"iss\'e game} $\EF(\model_1, \model_2, w_1, w_2)$ is defined as follows. There are two players called \emph{Spoiler} and
\emph{Duplicator}. Duplicator immediately looses if $w_1$ and $w_2$ do not agree.
Otherwise, the game starts, with the players moving
alternatively. Spoiler always makes the first move in a turn of the game, starting by choosing in which model he will make a move. Let us set $s=1$ and $d=2$ in case he chooses
$\model_1$; otherwise, let $s=2$ and $d=1$.

For the logics including the operators
$\{\tup{r},\remember,\known\}$, that is, $\cMLRK$ and $\cMLRKE$, the
possible moves are as follows:

\begin{enumerate}
\item Make a \emph{memorizing step}. I.e.,
Spoiler extends $S_s$ to $S_s \cup \{w_s\}$. The next turn then starts
with  $\EF(\model_1[w_1], \model_2[w_2],
w_1, w_2)$ (Duplicator does nothing in this case).

\item Make a \emph{move step}. I.e., Spoiler chooses $r \in \rel$, and $v_s$, an $R^s_r$-successor of $w_s$. If $w_s$ has no $R^s_r$-successors, then Duplicator wins. Duplicator has to chose $v_d$, an
$R^d_r$-successor of $w_d$, such that $v_s$ and $v_d$ agree. If there is
no such successor, Spoiler wins. Otherwise the game continues with
$\EF(\model_1, \model_2,
v_1, v_2)$.
\end{enumerate}

The moves for logics with the operators
$\{\ttup{r},\remember,\known\}$, that is, $\cMLRKM$ and $\cMLRKME$,
are similar, except that during a \emph{move step}, Spoiler always
remembers the current world, i.e., the game goes on with
$\EF(\model_1[w_1], \model_2[w_2], v_1, v_2)$ after Duplicator
response.


In the case of an infinite game, Duplicator wins. Note
that with this definition, exactly one of Spoiler or Duplicator wins
each game.

Given two pointed models $\tup{\gM_1,w_1}$ and $\tup{\gM_2,w_2}$ we
will write $\tup{\gM_1,w_1}$ $\equiv^{\EF}\tup{\gM_2,w_2}$ when
Duplicator has a winning strategy for $\EF(\gM_1, \gM_2, w_1, w_2)$
(the exact type of game involved will usually be clear by context,
and we will write $\equiv^{\EF}_\gL$ when we need to specifythat the game
corresponds to the language of the logic $\gL$).
\end{defn}

%An $n$\textit{-round} \textit{Ehrenfeucht game} (with $n \geq 0$) is
%defined with the same rules as the standard game, but Spoiler wins
%only when Duplicator cannot exhibit an appropriate successor at his
%$k$-th move, for some $k \leq n$. If Duplicator can respond in his
%$n$-th move, then he wins the game. We note an $n$-round game as
%$E_n(\model_1, \model_2, w_1, w_2)$. The game $E_0(\model_1,
%\model_2, w_1, w_2)$ is always won by Duplicator.

%\begin{defn}[$n$-Bisimulation]
%We say that two models $\model_1$ and $\model_2$ are $n$-bisimilar
%(and we write $\model_1 \bisim_n \model_2$) when there exists $w_1 \in \model_1$
%and $w_2 \in \model_2$ such that they agree and Duplicator has a
%winning strategy on $E_n(\model_1, \model_2, w_1, w_2)$. In this
%case we will also say that $w_1$  and $w_2$ are $n$-bisimilar
%($\model_1, w_1 \bisim_n \model_2, w_2$).
%\end{defn}

We now give an alternative definition of model equivalence using bisimulations.

\begin{defn}[Bisimulations]
Let $\gM_1$ and $\gM_2$ be two models. Let $\sim$ be a binary relation between $\wp(W_1)
\times W_1$ and $\wp(W_2) \times W_2$. So $\sim$ relates tuples
$\diam{\{m_1, m_2, \dots\}, m}$ with $\diam{\{n_1, n_2, \dots\},
n}$. We write these tuples as $\diam{M, m}$.

For the logics including the operators
$\{\tup{r},\remember,\known\}$ a \emph{bisimulation} should
satisfy the following properties:
\begin{center}
\begin{tabular}{rl}
(\emph{nontriv}) & $\sim$ is not empty.\\
(\emph{agree}) & If $\diam{M, m} \sim \diam{N, n}$, then $m$ and $n$ agree. \\

(\emph{forth}) & If $\diam{M, m} \sim \diam{N, n}$  and
$R_r^1(m,m')$, then there exists $n' \in W_2$\\
&  such that $R_r^2(n,n')$  and $\diam{M, m'} \sim \diam{N, n'}$.\\

(\emph{back}) & If $\diam{M, m} \sim \diam{N, n}$  and
$R_r^2(n,n')$, then there exists $m' \in W_1$ \\
&  such that $R_r^1(m,m')$  and $\diam{M, m'} \sim \diam{N, n'}$.\\

(\emph{remember}) & If $\diam{M, m} \sim \diam{N, n}$, then  $\diam{M \cup \{m\}, m} \sim \diam{N \cup \{n\}, n}$.
\end{tabular}
\end{center}

For logics with the operators $\{\ttup{r},\remember,\known\}$ we
slightly change (\emph{back}) and (\emph{forth}) to:
\begin{center}
\begin{tabular}{rl}
(\emph{mforth}) & If $\diam{M, m} \sim \diam{N, n}$  and
$R_r^1(m,m')$, then there exists $n' \in W_2$  such\\
& that $R_r^2(n,n')$  and $\diam{M \cup \{m\}, m'} \sim \diam{N \cup \{n\}, n'}$.\\

(\emph{mback}) & If $\diam{M, m} \sim \diam{N, n}$  and
$R_r^2(n,n')$, then there exists $m' \in W_1$  such\\
& that $R_r^1(m,m')$  and $\diam{M \cup \{m\}, m'} \sim \diam{N \cup \{n\}, n'}$.
\end{tabular}
\end{center}

Given two pointed models $\tup{\gM_1,w_1}$ and $\tup{\gM_2,w_2}$ we
write $\tup{\gM_1,w_1} \bisim \tup{\gM_2,w_2}$ if there is a
bisimulation linking $\tup{S_1,w_1}$ and $\tup{S_2,w_2}$ (again, the
exact type of bisimulation involved will usually be clear from the
context, and we will write $\bisim_\gL$ when we need to specify that the
bisimulation corresponds to the language of the logic $\gL$).
\end{defn}

The notions of
Ehrenfeucht-Fra\"iss\'e games and bisimulations for $\mathcal{L}_i,
i \in \linebreak \{1,2,3,4\}$ coincide, as indicated in the following theorem.

\begin{thm}\label{thm:equiv-g-s}
Given two pointed models $\tup{\model_1,w_1}$ and
$\tup{\model_2,w_2}$ then $\tup{\model_1,w_1}$ $\equiv^{EF}_\gL \tup{\model_2,w_2}$
if and only if $\tup{\model_1,w_1} \bisim_\gL \tup{\model_2,w_2}$.
\end{thm}

% \begin{pf}
% For the right to left direction. Assume that $\diam{S_1,w_1} \sim
% \diam{S_2, w_2}$ is an s-bisimulation. We will prove that there is a
% strategy for Duplicator in the game $\game$. First note that the
% game $\game$ is well defined, since by \textit{(prop)} and
% \textit{(known)}, $w_1$ and $w_2$ are agreeing states. We show that
% there is a strategy for Duplicator by proving that $(1)$ for any
% pair of tuples $\diam{S,w}$ and $\diam{Q,v}$ such that $\diam{S,w}
% \sim \diam{Q,v}$, and for any Spoiler step in the game
% $E(\model_1[S], \model_2[Q], w, v)$, there is always an appropriate
% answer for Duplicator such that the next step of the game is
% $E(\model_1[S'], \model_2[Q'], w', v')$ and $\diam{S',w'} \sim
% \diam{Q',v'}$. Given the initial assumptions, the fact that
% Duplicator has a winning strategy on the game $\game$ easily follows
% from $(1)$. So let's suppose that $\diam{S,w} \sim \diam{Q,v}$ and
% consider the game $E(\model_1[S], \model_2[Q], w, v)$. Without loss
% of generality, we assume that Spoiler chooses $\model_1$ to make his
% move. There are three kinds of moves Spoiler can do:
% \begin{itemize}
%  \item Spoiler chooses to make a memorizing step, so the game goes on with $E(\model_1[S \cup \{w\}], \model_2[Q \cup \{v\}],w,v)$. By the $(\remember)$ condition, we know that $\diam{S \cup \{w\}, w} \sim \diam{Q \cup \{v\}, v}$.
% \item Spoiler chooses to make a move step, so he chooses an $R$-succesor $w'$ of $w$. By the \textit{(forth)} condition (we should use \textit{(back)} here if we would have supposed that Duplicator chooses $\model_2$ to make his move), there is an $R$-succesor $v'$ of $v$ such that $\diam{S,w'} \sim \diam{Q,v'}$. Using \textit{(prop)} and \textit{(known)}, we know that $w'$ and $v'$ agree, so $v'$ is a good choice for Duplicator, the next step of the game is $E(\model_1[S], \model_2[Q], w', v')$ and $\diam{S,w'} \sim \diam{Q,v'}$.
% %\item Spoiler chooses to make a jump step, so he chooses any named state $w'$ of $\model_1$. Because $w'$ is named, there is a nominal $i$ that names $w'$. Because the signature is the same for both models, we know there is a $v'$ in $\model_2$ such that $i$ names $v'$. By the $(@)$ condition, we know that $\diam{S,w'} \sim \diam{Q,v'}$, so $v'$ is a good choice for Duplicator, because by \textit{(prop)} and \textit{(known)} $w'$ and $v'$ agree. The next step of the game is $E(\model_1[S], \model_2[Q], w', v')$ and $\diam{S,w'} \sim \diam{Q,v'}$.
% \end{itemize}
%
% For the other direction, let's suppose that Duplicator has a winning
% strategy $\mathcal{S}$ on the game $\game$, where
% $\model_1=\diam{W_1,R_1,S_1,V_1}$ and
% $\model_2=\diam{W_2,R_2,S_2,V_2}$. We are going to define an
% s-bisimulation relation $\sim$ in the following way: $\diam{S,w} \sim
% \diam{Q,v}$ iff $E(\model_1[S], \model_2[Q],w,v)$ is some stage of
% the game $\game$ reached by the players when Duplicator follows the
% strategy $\mathcal{S}$.
%
% So now we have to see that the relation $\sim$ we defined is
% actually an s-bisimulation. Suppose that $\diam{S,w} \sim \diam{Q,v}$.
% Using the definition of $\sim$, that means that $E(\model_1[S],
% \model_2[Q],w,v)$ is a reacheable game state from $\game$ when
% Duplicator uses the strategy $\mathcal{S}$. Let's check the
% bisimulation conditions:
% \begin{itemize}
%  \item The conditions \textit{(prop)} and \textit{(known)} are easy to see, given that the definition of $\sim$ implies that if $\diam{S,w} \sim \diam{Q,v}$ then $w$ and $v$ are agreeing states.
% \item To see that the \textit{(forth)} condition holds, suppose that $R^\model(m,m')$. One possible move for Spoiler in the game $E(\model_1[S], \model_2[Q],w,v)$ is to choose $m'$ from $\model_1$, and because Duplicator uses the winning strategy $\mathcal{S}$, he can answer with a state $v' \in \model_2$, a succesor of $v$, such that $w'$ and $v'$ agree. Therefore, the next step of the game is $E(\model_1[S], \model_2[Q], w', v')$, and by definition, $\diam{S,w'} \sim \diam{Q,v'}$. The \textit{(back)} condition is equivalent.
% %\item To see $(@)$, note that for every nominal $i \in \nom$, Spoiler can make a jump state and move to $w'$, a state named by $i$ (and without losing generality we can suppose that $w' \in \model_1$). Using $\mathcal{S}$, Duplicator can answer with a state $v' \in \model_2$, such that $w'$ and $v'$ agree, and the next step of the game is $E(\model_1[S], \model_2[Q], w', v')$. Therefore $\diam{S,w'} \sim \diam{Q,v'}$ by definition.
% \item Finally, to see the $(\remember)$ condition, note that in the game $E(\model_1[S], \model_2[Q],w,v)$ Spoiler can choose to make a memorizing step, and therefore the next step of the game is $E(\model_1[S \cup \{w\}], \model_2[Q \cup \{v\}],w,v)$. By definition, that means that $\diam{S\cup\{w\},w} \sim \diam{Q \cup \{v\},v}$.
% \end{itemize}
% Therefore, $\sim$ is actually an s-bisimulation. Because the game
% $\game$ is a (trivial) reachable stage of itself, $\diam{S_1,w} \sim
% \diam{S_2,v}$ as desired.
% \end{pf}

% We now define what bisimulation between two models is. Of course, given Theorem~\ref{thm:equiv-g-s},
% we can equivalently use the game or structural notions.
%
% \begin{defn}[Bisimulation]
% We say that two models $\model_1$ and $\model_2$ are {\em
% bisimilar} (and we write $\model_1 \bisim \model_2$) iff
% there exist g-bisimilar $w_1 \in \model_1$ and $w_2 \in \model_2$.
% In this case we also say that $w_1$  and $w_2$ are {\em
% bisimilar} ($\model_1, w_1 \bisim \model_2, w_2$).
% \end{defn}
%

%%%%%%%%%%%%%%%%%%%%%%%%%%%%%%%%%%%%%%%%%%%%%%%%%%%%%%%%%%%
%%%%%%%%%%%%%%%%%%%%%%%%%%%%%%%%%%%%%%%%%%%%%%%%%%%%%%%%%%%
%%%%%%%%%%%%%%%%%%%%%%%%%%%%%%%%%%%%%%%%%%%%%%%%%%%%%%%%%%%

Moreover, both notions of model equivalence preserve
the truth value of formulas. Given two pointed models
$\tup{\model_1,w_1}$ and $\tup{\model_2,w_2}$, we write
$\tup{\model_1,w_1} \linebreak \equiv_\mathcal{L} \tup{\model_2,w_2}$ if for
any formula $\varphi$ in the language of the logic $\mathcal{L}$
we have that $\gM_1,w_1 \models \varphi$ iff and only if $\gM_2,w_2
\models \varphi$.


\begin{thm}\label{bisim}
Let $\tup{\model_1, w_1}$ and $\tup{\model_2,w_2}$ be two pointed
models. Then $\tup{\gM_1,w_1}$ $\bisim_\gL \tup{\gM_2,w_2}$ ($\tup{\gM_1,w_1}$ $\equiv^\EF_\gL \tup{\gM_2,w_2}$) implies
$\tup{\gM_1,w_1} \equiv_\gL \tup{\gM_2,w_2}$.

If
$\model_1$ and $\model_2$ are image finite (i.e., each element of the
domain has only a finite number of successors by each accessibility
relation). Then
$\tup{\gM_1, w_1} \equiv_\gL \tup{\gM_2, w_2}$  implies
$\tup{\gM_1, w_1} \bisim_\gL \tup{\gM_2, w_2}$ ($\tup{\gM_1, w_1} \equiv^\EF_\gL \tup{\gM_2, w_2}$).
\end{thm}



% \tb{The proof below is standard and can be dropped}
%
% \begin{pf}
% We prove that if $w_1$ and $w_2$ agree and Duplicator has a winning
% strategy on $E(\model_1, \model_2, w_1, w_2)$ then $\forall \varphi
% \in \tl, \, \model_1, w_1 \models \varphi \; \iff \; \model_2, w_2
% \models \varphi$. We proceed by induction on $\varphi$.
%
% \begin{itemize}
% \item The propositional and boolean cases are trivial.
%
% \item $\varphi = \known$. This case follows from Definition~\ref{semantics}
% and because $w_1$ and $w_2$ agree.
%
% \item $\varphi = \diam{r} \psi$. This is the standard modal case.
% Preservation is ensured thanks to the \emph{move steps} in the
% definition of the game.
% %We prove that $\model_1,w_1 \models
% %\diam{R} \psi$ implies $\model_2,w_2 \models \diam{R} \psi$. Suppose
% %$\model_1,w_1 \models \diam{R} \psi$. Then there is $v_1\in
% %\model_1$ such that $w_1R_1v_1$ and $\model_1,v_1\models\psi$.
% %Suppose that Spoiler chooses model $\model_1$ at step 1, not to
% %remember the current world at step 2 and $v_1$ as an $R_1$-successor
% %at step 3. Following Duplicator's winning strategy on $E(\model_1,
% %\model_2, w_1, w_2)$, he finds $v_2$ agreeing with $v_1$,
% %$w_2R_2v_2$ and keeps a winning strategy on $E(\model_1, \model_2,
% %v_1, v_2)$. By inductive hypothesis and the fact that
% %$\model_1,v_1\models\psi$, we conclude $\model_2,v_2\models\psi$ and
% %therefore $\model_2,w_2\models\diam{R} \psi$. The other direction is
% %identical.
%
% \item $\varphi = \remember \psi$. We prove that
% $\model_1,w_1 \models \remember \psi$ implies $\model_2,w_2 \models
% \remember \psi$. Suppose $\model_1,w_1 \models \remember \psi$ then
% $\model_1[w_1],w_1 \models \psi$. The following claim is clear.
%
% \bigskip
%
% \noindent
% \textbf{Claim.} Let $\model_1,\model_2$ be two models, $w_1 \in \model_1,w_2
% \in \model_2$. If Duplicator has a winning
% strategy on $\game$ then he has a winning strategy on
% $E(\model_1[w_1], \model_2[w_2], w_1, w_2)$.
%
% \bigskip
%
%
% By this claim,
% Duplicator has a winning strategy on $E(\model_1[w_1],
% \model_2[w_2], w_1,$ $w_2)$. Applying inductive hypothesis and the
% fact that $\model_1[w_1],w_1 \models \psi$, we conclude
% $\model_2[w_2], w_2 \models \psi$  and then $\model_2, w_2 \models
% \remember\psi$. The other direction is identical.
% \end{itemize}
% This concludes the proof.
% \end{pf}

% \begin{thm}\label{thm:bisim-n-to-modal-n}
% Let $\model_1,\model_2$ be two models, $w_1 \in \model_1, w_2 \in
% \model_2$. If $w_1 \bisim_n w_2$ then $w_1 \leftrightsquigarrow_n
% w_2$.
% \end{thm}
%
% \begin{pf}
% The proof can be done by induction on $n$, following the same idea
% as in Theorem~\ref{bisim}.
% \end{pf}
%
% \begin{thm}\label{thm:bisim-n-to-prune}
% Let $\model$ be a model and $w \in \model$. For every $n \geq 0$,
% $\model, w \bisim_n (\model \upharpoonright n), w$.
% \end{thm}
%
% \begin{pf}
% It is easy to see that in an $n$-round game the players only visit
% states with height at most $n$ and thus, as far as the players are
% concerned, $\model$ and $\model\upharpoonright n$ are the same.
% Therefore Duplicator can always copy Spoiler's move.
% \end{pf}

% The converse of Theorem~\ref{bisim} holds for image-finite models
% (i.e., models in which the set of successors
% of any state in the domain is finite). The proof is exactly the
% same as for $\mathcal{K}$, as $\remember$ and $\known$ do not
% interact with the accessibility relation~\cite{BRV01}.


%\begin{thm}[Hennessy-Milner Theorem]\label{thm:hennesy}
%\end{thm}

%\begin{pf}
%We prove that the relation
%$\leftrightsquigarrow$ of modal equivalence induces at any stage of the game, the next step in Duplicator's strategy. We prove that for any $\model_1', \model_2', w_1'\in\model_1', w_2'\in\model_2'$ such that $\model_1',w_1' \leftrightsquigarrow \model_2',w_2'$, and for any $v_1\in\model_1'$ that Spoiler selects, Duplicator can always answer $v_2\in\model_2'$ such that $\model_1',v_1\leftrightsquigarrow \model_2',v_2$ (and symmetrically in the case Spoiler chooses $\model_2'$). Suppose
%Spoiler chooses $\model_1'$ to play.
%\begin{enumerate}
%item Suppose Spoiler decides to take a \emph{move step}.  He
%chooses $v_1$, an $R$-successor of $w_1'$. Assume by contradiction
%that Duplicator cannot respond back. This means that he cannot
%exhibit $v_2$, an $R$-successor of $w_2'$, such that
%$\model_1',v_1 \leftrightsquigarrow \model_2',v_2$.
%Suppose $\{u_1, \dots, u_n\}$ are the successors of $w_2'$. Let
%$\psi_1, \dots, \psi_n$ be formulas such that $\model_2', u_i
%\not\models \psi_i$ and $\model_1',v_1 \models \psi_i$. It follows
%that $\model_1', w_1' \models \diam{R}(\psi_1 \land \dots \land
%\psi_n)$ and $\model_2', w_2' \not \models \diam{R}(\psi_1 \land
%\dots \land \psi_n)$.  This is a contradiction, since we supposed
%$\model_1',w_1' \leftrightsquigarrow \model_2',w_2'$.

%\item Suppose Spoiler decides to remember $w_1'$. As in the above
%case, there are formulas $\psi_1,\dots,\psi_n$ such that
%$\model_1'[w'_1], w_1' \models \diam{R}(\psi_1 \land \dots \land
%\psi_n)$ and $\model_2'[w_2'], w_2' \not \models \diam{R}(\psi_1
%\land \dots \land \psi_n)$. By definition, $\model_1', w_1' \models
%\remember\diam{R}(\psi_1 \land \dots \land \psi_n)$ and $\model_2',
%w_2' \not \models \remember\diam{R}(\psi_1 \land \dots \land
%\psi_n)$ and we arrive to the same contradiction as in the previous
%case.
%\end{enumerate}
%This concludes the proof.
%\end{pf}

%\begin{defn}[Syntax]\label{syntax}
%Let $\prop=\{p_1, p_2, \dots\}$ (the \textit{propositional symbols})
%and $\rel=\{r_1, r_2, \dots\}$ (the \textit{relational symbols}) be
%pairwise disjoint, countable infinite sets of symbols. The set
%$\forms$ of formulas of $\tlm$ in the signature $\langle
%\prop,\rel\rangle$ is defined as:
%$$
%\forms ::= p \mid \known \mid \lnot \varphi \mid \varphi_1 \land
%\varphi_2 \mid \diam{R} \varphi \mid \remember \varphi,
%$$
%where $p \in \prop$, $R \in \rel$  and $\varphi, \varphi_1,
%\varphi_2 \in \forms$.
%\end{defn}





%We define bisimulation and $n$-bisimulation between models
%($\model_1 \bisim \model_2$ and $\model_1 \bisim_n \model_2$) in the
%same way as before. In particular, note that Theorem~\ref{bisim} and
%\ref{thm:hennesy} hold for $\tlm$.
%
%SACAR LOS TEOREMAS Y SOLO HACER REFERENCIA A LOS QUE YA TENEMOS
%
%\begin{thm}
%Let $\model_1, \model_2$ be two models, $w_1 \in \model_1, w_2 \in
%\model_2$. If $w_1 \bisim w_2$ then $w_1 \leftrightsquigarrow w_2$.
%\end{thm}
%
%\begin{thm}
%Let $\model_1, \model_2$ be two image finite models. Then for every
%$w_1 \in \model_1$ and $w_2 \in \model_2$,  $w_1 \bisim w_2$ iff
%$w_1 \leftrightsquigarrow w_2$.
%\end{thm}


%\begin{defn}[$n$-Bisimulation]
%We say that two models $\model_1$ and $\model_2$ are $n$-bisimilar
%($\model_1 \bisim_n \model_2$) when there exists $w_1 \in \model_1$
%and $w_2 \in \model_2$ such that they agree in the same
%propositional symbols and Duplicator has a winning strategy on
%$E_n(\model_1, \model_2, w_1, w_2)$. In this case we will also say
%that $w_1$  and $w_2$ are $n$-bisimilar ($\model_1, w_1 \bisim_n
%\model_2, w_2$).
%\end{defn}
%
%FALTA DEFINIR $E_n$ Y $EQUIV_n$
%
%\begin{thm}\label{thm:bisim-n-to-modal-n}
%Let $\model_1,\model_2$ be two models, $w_1 \in \model_1, w_2 \in
%\model_2$. If $w_1 \bisim_n w_2$ then $w_1 \leftrightsquigarrow_n
%w_2$.
%\end{thm}
%
%\begin{pf}
%The proof can be done by induction on $n$.
%\end{pf}
%
%For a natural number $k$, the restriction of a model $\model$ to $k$
%(notation: $\model \upharpoonright k$) is defined as the submodel
%containing only states whose height is at most $k$.
%
%
%\begin{thm}\label{thm:bisim-n-to-prune}
%Let $\model$ be a model and $w \in \model$. For every $n \geq 0$,
%$\model, w \bisim_n (\model \upharpoonright n), w$.
%\end{thm}
%
%\begin{pf}
%It is easy to see that in an $n$-round game the players only visit
%states with height at most $n$ and thus, as far as the players are
%concerned, $\model$ and $\model\upharpoonright n$ are the same.
%Therefore Duplicator can always copy Spoiler's moves.
%\end{pf}
%
%
%% \begin{pf}
%% We follow the same idea as in Theorem~\ref{bisim}. The only non
%% trivial case is $\varphi = \diam{R} \psi$. We prove that
%% $\model_1,w_1 \models \diam{R} \psi$ implies $\model_2,w_2 \models
%% \diam{R} \psi$. Suppose $\model_1,w_1 \models \diam{R} \psi$. Then
%% there is $v_1\in \model_1$ such that $w_1R_1v_1$ and
%% $\model_1[w_1],v_1\models\psi$. Suppose that Spoiler chooses model
%% $\model_1$ at step 1 and $v_1$ as an $R_1$-successor. By
%% Lemma~\ref{lem:strategy}, Duplicator has a winning winning strategy
%% on $E(\model_1[w_1], \model_2[w_2], w_1, w_2)$, and following that
%% strategy, he finds $v_2$ such that $\props(v_1)=\props(v_2)$,
%% $w_2R_2v_2$, and keeps a winning strategy on $E(\model_1[w_1],
%% \model_2[w_2], v_1, v_2)$. By inductive hypothesis and the fact that
%% $\model_1[w_1],v_1\models\psi$, we conclude
%% $\model_2[w_2],v_2\models\psi$ and therefore
%% $\model_2,w_2\models\diam{R} \psi$. The other direction is
%% identical.
%% \end{pf}
