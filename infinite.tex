\section{Infinite Models and Undecidability}

In this section we will analyze the decidability and the finite
model property of the memory logics we introduced. First, we will
show that $\cMLRKM$ has the bounded model property and, therefore,
it is a decidable logic. On the other hand, the languages $\cMLRK$,
$\cMLRKE$ and $\cMLRKME$ lack the finite model property and they are
undecidable. We also analyze the case when $\cMLRKM$ is extended
with only one nominal $i$, and we show that $\cMLRKM + i$ is also
undecidable and that an infinite model can be forced.

\tb{Estaría bueno reescribir un poco el párrafo de arriba, como para agregar alguna intuición sobre por qué sucede esto, y que de alguna manera la decidibilidad de L3 no es muy ``robusta''}


\subsection{A decidable memory logic}

We start by establishing an equivalence result between $\bml$ and $\cMLRKM$ for
the class of tree-like models. We first define, given a formula
$\varphi$, a new formula $\varphi^\sharp$, as the result of
replacing all the ocurrences of $\known$ that are in $\varphi$ at
modal depth zero by $\top$. More formally:

\begin{displaymath}\label{sharp}
\begin{array}{rcl}
p^\sharp & = & p \quad p \in \prop\\
\known^\sharp & = & \top \\
(\lnot \varphi)^\sharp & = & \lnot \varphi^\sharp \\
(\varphi_1 \land \varphi_2)^\sharp & = & \varphi_1^\sharp \land \varphi_2^\sharp \\
(\remember \varphi)^\sharp & = & \remember \varphi^\sharp\\
(\ttup{r} \varphi)^\sharp & = & \ttup{r} \varphi
\end{array}
\end{displaymath}

\begin{lem}\label{lem:replace}
$\model, w \models \remember \varphi$ iff $\model, w \models
\varphi^\sharp$.
\end{lem}

\begin{pf}
We proceed by induction. The case for $\known$, the propositional
symbols and booleans are straightforward. We analyze the other
cases:
\begin{itemize}
 \item $\varphi = \remember \psi$. $\model, w \models \remember\remember\psi$ iff $\model, w \models \remember \psi$ iff (by inductive hypothesis) $\model, w \models \psi^\sharp$ iff $\model, w \models (\psi^\sharp)^\sharp$ iff (by inductive hypothesis) $\model, w \models \remember (\psi^\sharp)$ iff $\model, w \models (\remember\psi)^\sharp$
\item $\varphi = \ttup{r} \psi$. $\model, w \models \remember \ttup{r} \psi$ iff (by definition) $\model[w], w \models \ttup{r} \psi$ iff (by definition of $\sharp$) $\model[w], w \models (\ttup{r} \psi)^\sharp$ iff (by definition) $\model, w \models (\ttup{r} \psi)^\sharp$
\end{itemize}
\end{pf}

We are now ready to define an equivalence preserving translation for
the tree-like models.

\begin{defn}[Translation]\label{def:tr-tlm-k}
Let $known$ be a new proposition symbol not in $\prop$. The
translation $\Tr$, taking $\cMLRKM$ formulas over the signature
$\diam{\prop, \rel}$ to $\bml$-formulas over the signature
$\diam{\prop \cup \{known\}, \rel}$ is defined as:

\begin{displaymath}
\begin{array}{rcl}
\Tr(p) & = & p \quad p \in \prop\\
\Tr(\known) & = & known \\
\Tr(\lnot \varphi) & = & \lnot \Tr(\varphi) \\
\Tr(\varphi_1 \land \varphi_2) & = & \Tr(\varphi_1) \land \Tr(\varphi_2) \\
\Tr(\ttup{r} \varphi) & = & \diam{r} \Tr(\varphi) \\
\Tr(\remember \varphi) & = & \Tr(\varphi^\sharp)
\end{array}
\end{displaymath}

\end{defn}

Taking the same model equivalence used to compare $\bml$ with $\cMLRK$,
the following result can be established:

\begin{pro}[Satisfiability preserving]\label{prop:sat-preserv-tree}
Let $\model$ be a tree-like model. Then
$\model,w\models_{\cMLRK}\varphi$ iff $\model,w\models_{\bml}
\Tr(\varphi)$.
\end{pro}

\begin{pf}
We proceed by induction on $\varphi$. The propositional and boolean
cases are trivial. The $\known$ case is also easy given the
definitions. For the diamond case. Given that $\model$ is tree-like,
the remember operator has no effect beyond modal operators, so
$\model, w \models_{\cMLRKM} \ttup{r}\psi$ iff exists $v$ such that
$wRv$ and $\model,v\models_{\cMLRKM} \psi$. By inductive hypothesis,
$\model,v\models_{\cMLRKM} \psi$ iff $\model, v \models_{\bml}
\Tr(\psi)$, and by definition $\model, w \models_{\bml} \diam{r}
\Tr(\psi)$. Finally, let's see the case for remember. By
Lemma~\ref{lem:replace}, $\model, w \models_{\cMLRKM} \remember \psi$
iff $\model, w \models_{\cMLRKM} \psi^\sharp$. By inductive hypothesis,
$\model, w \models_{\bml} \Tr(\psi^\sharp)$.
\end{pf}



%\subsection{Finite model property}

% \begin{lem}\label{lem:frontera-k}
% Let $\model$ be a model and $w \in \model$. Then $(\model
% \upharpoonright n),w \bisim_n \model, w$.
% \end{lem}
%
% \begin{pf}
% In the game $E_n(\model, \model \upharpoonright n, w, w)$ the
% winning strategy for Duplicator consists in copying Spoiler moves.
% This can always be done because Spoiler cannot go beyond $n$ steps
% from $w$.
% \end{pf}

% \begin{thm}
% Let $\model$ be a model and $w \in \model$. For every $n \geq 0$,
% $\model, w \leftrightsquigarrow_n (\model \upharpoonright n), w$.
% \end{thm}
%
% \begin{pf}
% This result follows from Theorem~\ref{thm:bisim-n-to-prune} and
% Theorem~\ref{thm:bisim-tlm-n-to-modal-n}.
% \end{pf}



\begin{pro}[Tree model
property]\label{prop:tree-model-property} For every $\cMLRKM$-model
$\model$ and $w\in\model$ there is a tree-like model $\model'$ where
$w\in\model'$ and $\model,w\bisim\model',w$.
\end{pro}

\begin{pf}
Let $\model=\diam{W,R,V,S}$.  We prove the result for the unimodal
case, the generalization to the multimodal case is straightforward.
We define the model $\model'=\diam{W',R',V',S'}$ as follows. Its
domain $W'$ consists on all finite sequences $(u_0,\dots,u_n)$ such
that $u_0=w$, $n\geq 0$ and there is a path $u_0Ru_1\dots Ru_n$ in
$\model$. Define $(u_0,\dots,u_n)R(v_0,\dots,v_m)$ to hold if
$m=n+1$, $u_i=v_1$ for $i=0,\dots,n$ and $u_nRv_m$ holds in
$\model$. The valuation $V'$ is defined by setting
$(u_0,\dots,u_n)\in V'(p)$ iff $u_n\in V(p)$. Finally,
$(u_0,\dots,u_{n-1},u_n)\in S'$ iff $u_n\in\{u_0,\dots,u_{n-1}\}$ or
$u_n\in S$.

We call $s_i$ the sequence $(v_0,\dots,v_i)$ for every
$i=0,\dots,n$. Let us see that Duplicator has a winning strategy in
the game $\EF(\model,\model',w,w)$. It is sufficient to see that in
the game
$\EF(\model[v_0,\dots,v_n],\model'[s_0,\dots,s_n],v_{n+1},s_{n+1})$,
Duplicator can always answer successfully to Spoiler's move.
\begin{itemize}
\item
If Spoiler chooses $\model[v_0,\dots,v_n]$ and some $v_{n+1}$, a
successor of $v_n$, Duplicator chooses the sequence
$s_{n+1}=s_nv_{n+1}$.

\item
If Spoiler chooses $\model'[s_0,\dots,s_n]$ and $s_{n+1}=s_nv_{n+1}$
(for some $v_{n+1}$), a successor of $s_n$, Duplicator chooses the
state $v_{n+1}$.
\end{itemize}
By definition, it is easy to see that $s_{n+1}$ and $v_{n+1}$ agree.
Observe that at this stage of the game the current ``memory" of
$\model[v_0,\dots,v_n]$ is $S \cup \{v_0,\dots,v_n\}$ and the
current ``memory" of $\model'[s_0,\dots,s_n]$ is $S' \cup
\{s_0,\dots,s_n\}$. Then it is also clear that $v_{n+1}$ was
remembered at this stage of the game, if and only if $s_{n+1}$ also
was. More formally, on the one hand, $v_{n+1}\in S \cup
\{v_0,\dots,v_n\}$ implies $s_{n+1} \in S'$ by definition. On the
other hand if $s_{n+1}\in S' \cup \{s_0,\dots,s_n\}$, we have
$s_{n+1}\in S'$ (since there are no cycles in $\model'$) and by
definition $v_{n+1}\in S \cup \{v_0,\dots,v_n\}$.
\end{pf}

%If $\model$ is a $\cMLRKM$-model, we denote with $\model_T$ the
%tree-like model constructed in the above proposition.

%\begin{lem}\label{lem:bisim_bml_then_bisim_tlm}
%Let $\model,\nodel$ be two $\tlm$ tree-like models. For any $n$, if
%the equivalent $\bml$-models satisfy
%$\model,w\bisim_n^{\bml}\nodel,v$ then
%$\model,w\bisim_n^{\tlm}\nodel,v$.
%\end{lem}
%
%\begin{pf}
%Let $\model=\diam{W_1,R_1,V_1,S_1}$ and
%$\nodel=\diam{W_2,R_2,V_2,S_2}$.
%%
%Recall the definition of $n$-bisimulation for $\bml$
%\cite[Definition 2.30]{BRV01}, involving the relations
%$Z_0\supseteq\dots\supseteq Z_{n}$. We define a winning strategy for
%Duplicator in terms of $Z_0,\dots,Z_{n}$.
%
%We start in the game $E(\model,\nodel,w,v)$. We show that if
%$E(\model',\nodel',w',v')$ is some $i$th stage of that game ($i\leq
%n$) and $w'Z_{n-i}v'$, then for any choice $u$ of Spoiler,
%Duplicator can always answer with $t$ such that $uZ_{n-i-1}t$.
%Suppose Spoiler chooses $u\in\model'$ such that $w'Ru$. By
%definition of $Z_{n-i}$, there exists $t\in\nodel'$ such that $v'Rt$
%and $vZ_{n-i-1}t$. It follows from the definition of $Z_0$ that $u$
%and $t$ agree. Observe that the game $E(\model',\nodel',w',v')$ is
%just equivalent to $E(\model,\nodel,w',v')$, since both $\model$ and
%$\nodel$ are acyclic. Therefore $u\in S_1$ iff $t\in S_2$. The case
%Spoiler chooses to play in $\nodel'$ is similar and this completes
%the proof.
%\end{pf}

%\begin{thm}\label{thm:fmp-tlm}
%Let $\model$ be a model. For every $n$ there is a finite tree-like
%model $\nodel$ such that $\model\bisim_n\nodel$.
%\end{thm}

%\begin{pf}
%By Proposition~\ref{prop:tree-model-property},
%$\model\bisim^{\tlm}\model_T$. By
%Theorem~\ref{thm:bisim-n-to-prune},
%$\model_T\bisim_n^{\tlm}\model_T\upharpoonright n$. Applying the
%selection process in \cite[Theorem 2.34]{BRV01} on the equivalent
%$\bml$-model for $\model_T\upharpoonright n$ we obtain  a finite
%tree-like model $\nodel$ such that $\model_T\upharpoonright
%n\bisim_n^{\bml}\nodel$. Applying
%Lemma~\ref{lem:bisim_bml_then_bisim_tlm}, we know that
%$\model_T\upharpoonright n\bisim_n^{\tlm}\nodel$ and therefore
%$\model \bisim_n^{\tlm}\nodel$.
%\end{pf}

\begin{thm}\label{thm:fmp}
If $\varphi\in\cMLRKM$ is satisfiable then it is satisfiable on a
recursively bounded model.
\end{thm}

\begin{pf}
Let $\varphi$ be a $\cMLRKM$ formula of degree $k$, and suppose
$\model,w \models_{\cMLRKM} \varphi$. By
Proposition~\ref{prop:tree-model-property}, there is a tree-like
model $\model'$ such that $\model',w \models_{\cMLRKM} \varphi$.
Using Proposition~\ref{prop:sat-preserv-tree}, we can turn to basic
modal logic $\bml$, and assert that $\model',w \models_{\bml}
\Tr(\varphi)$. Now we can use the finite tree model property for
basic modal logic~\cite{BRV01}, so there must be a finite tree-like
model $\model''$ such that $\model'',w \models_{\bml} \Tr(\varphi)$.
Finally, we can use Proposition~\ref{prop:sat-preserv-tree} again,
and conclude $\model'',w \models_{\cMLRKM} \varphi$.

Observe that it is known~\cite{BRV01} that $\model''$ has at most
$n^k$ states, where $n$ is the number of non-equivalent formulas of
degree at most $k$, containing the same propositional symbols of
$\varphi$.
\end{pf}


%\begin{cor}[Finite model property]
%If $\varphi\in\tlm$ is satisfiable then it is satisfiable on a
%finite model.
%\end{cor}

%\begin{thm}[Alternative proof for the finite model property] (without using Lemma~\ref{lem:bisim_bml_then_bisim_tlm})
%If $\varphi\in\tlm$ is satisfiable then it is satisfiable on a
%finite model.
%\end{thm}
%\begin{pf}
%Let $\varphi$ be a $\tlm$ formula, and suppose $\model,w \models
%\varphi$. By Proposition~\ref{prop:tree-model-property}, there is a
%tree-like model $\model'$ such that $\model',w \models \varphi$.
%Using Proposition~\ref{prop:sat-preserv-tree}, we can turn to basic
%modal logic $\bml$, and assert that $\model',w \models_{\bml}
%Tr(\varphi)$. Now we can use the finite tree model property for
%basic modal logic \cite{??}, so there must be a finite tree-like
%model $\model''$ such that $\model'',w \models_{\bml} Tr(\varphi)$.
%Finally, we can use Proposition~\ref{prop:sat-preserv-tree} again,
%and conclude $\model'',w \models \varphi$.
%\end{pf}


\begin{cor}[Decidability]\label{cor:tlm-decidability}
$\cMLRKM$ is decidable.
\end{cor}

\begin{pf}
By Theorem~\ref{thm:fmp}, given a $\cMLRKM$-formula $\varphi$ one
can recursively obtain a bound $b$ such that if $\varphi$ is
satisfiable then it is satisfiable in a model of at most $b$ states.
Hence, to decide the satisfiability of $\varphi$ one can check if
$\model\models\varphi$ for all models $\model$ with size at most
$b$.
\end{pf}

\begin{thm}[\mbox{\rm\textsc{pspace}}-complete]
The satisfaction problem for $\cMLRKM$ is \textsc{pspace}-complete.
\end{thm}
\begin{pf}
The lower \textsc{pspace}-hard bound for $\cMLRKM$ follows from the
lower bound for the basic modal logic~\cite{BRV01}. We show a
matching upper bound by reducing the satisfiability problem for
basic modal logic. Given a $\cMLRKM$-formula $\varphi$, we compute
$\Tr(\varphi)$ using Definition~\ref{def:tr-tlm-k}. Note that the
size of the translated formula $\Tr(\varphi)$ is actually at most
the size of $\varphi$. Let suppose that $\Tr(\varphi)$ has a model,
that is $\model, w \models_{\bml} \Tr(\varphi)$. This happens iff
there exists a tree-like model $\model'$ such that $\model', w
\models_{\bml} \Tr(\varphi)$ (by the tree model property of $\bml$)
iff $\model', w \models_{\cMLRKM} \varphi$ (by
Proposition~\ref{prop:sat-preserv-tree}) iff there exists a model
$\nodel$ such that $\nodel,w \models_{\cMLRKM} \varphi$ (by the tree
model property for $\cMLRKM$, stated in
Proposition~\ref{prop:tree-model-property}). Therefore, given
$\varphi$, determining the $\bml$-satisfiability of $\Tr(\varphi)$
(and we can do this with a \textsc{pspace} algorithm~\cite{BRV01})
is enough to decide the $\cMLRKM$-satisfiability of $\varphi$.
\end{pf}



\subsection{Undecidable memory logics}

In this subsection we show that the rest of the memory logics
presented in this work are undecidable. We start by analyzing the
decidability of two variations of $\cMLRKM$.

We first show that adding just one nominal $i$ to $\cMLRKM$ is
sufficient to turn into undecidability. We denote with $\cMLRKM+i$
this extension of $\cMLRKM$.
%
We next show that the satisfiability problem of the logic $\cMLRKM$
restricted to the class of models with $S=\emptyset$ (i.e.\ the
logic $\cMLRKME$) is also undecidable.

From the undecidability of $\cMLRKME$, we will conclude the
undecidability of $\cMLRK$ and $\cMLRKE$. Additionally, every time
we show undecidability, we also prove lack of finite model property.

\tb{SS: podemos decir que de alguna manera L3 no es muy robusta en
cuanto a decidibilidad}

%The last issue that we will discuss in this paper
%is the undecidability of the satisfiability problem
%for both $\cMLRK$ and $\cMLRKE$.  The proof is an adaptation
%of the proof of undecidability of $\hlogic$ presented in~\cite{BS95}.

%We first prove that both languages lack the finite model property~\cite{BRV01}.


\begin{thm}\label{thm:tlmi:inf}
$\cMLRKM + i$ lacks the finite model property.
% There is a formula
%$\textit{Inf} \in \cMLRKM + i$ such that $\mathcal{M},w \models
%\textit{Inf}$ implies that the domain of $\mathcal{M}$ is infinite.
\end{thm}

\begin{pf}
Consider the following formulas:
$$
\begin{array}{rl}
(\textit{Back}) & i \land \bbox{r}\lnot i  \land \ttup{r} \top \land \bbox{r}\ttup{r} i \\
%&$``The root is irreflexive, and in symmetric relation with all its (nonempty) successors.''$\\
(\textit{Empty}) & \bbox{r}\lnot \known \land \bbox{r}\bbox{r}(\lnot i \to \lnot \known)\\
%&$``One step and two steps successors are not known.''$\\
(\textit{Spy}) & \bbox{r}\bbox{r}( \lnot i \to \ttup{r} (i \land \ttup{r} ( \known \land \lnot \ttup{r} ( \known \land \lnot i))))\\
%&$``All successors are irreflexive.''$\\
%(\textit{Irr}) & \lnot \ttup{r} \ttup{r} ( \lnot i \land \known))\\
(\textit{Succ}) & \bbox{r}\ttup{r} \lnot i \\
(\textit{no-3cyc}) & \lnot \ttup{r} \ttup{r} ( \lnot \known \land \ttup{r} ( \lnot \known \land \ttup{r} ( \known \land \lnot i)))\\
(\textit{Tran}) & \bbox{r} \ttup{r} ( i \land \bbox{r} ( \lnot \known \to \ttup{r}( i \land \ttup{r} ( \known \land  \ttup{r}(\known \land \lnot i) ))))\\
\end{array}
$$
%
Let $\textit{Inf}$ be $\textit{Back} \land \textit{Empty} \land
\textit{Spy} \land \textit{Irr} \land \textit{Succ} \land
\textit{3cyc} \land \textit{Tran}$. Let $\model=\diam{W, R, V,
\emptyset}$. Let us see that if $\model, w \models \textit{Inf}$,
then $W$ is infinite.
%
Suppose $\model, w \models \textit{Inf}$. Let $B = \{b \in W \mid
wRb\}$. Because $\textit{Back}$ is satisfied, $w \not \in B$, $B
\not= \emptyset$ and for all $b \in B$, $bRw$. Note that
\textit{Empty} sais that the one and two-step horizon is not known,
and also this implies that every state in $B$ is irreflexive.
Because $\textit{Spy}$ is satisfied, if $a \not= w$ and $a$ is a
successor of an element of $B$ then $a$ is also an element of $B$.
As $\textit{Succ}$ is satisfied at $w$, every point in $B$ has a
successor distinct from $w$. As $\textit{no-3cyc}$ is satisfied,
there cannot be $3$ different elements in $B$ forming a cycle, and
this sentence together with $\textit{Tran}$ force $R$ to
transitively order $B$.
%
It follows that $B$ is an unbounded strict partial order, hence
infinite, and so is $W$.
\end{pf}


We now show that this logic is undecidable by encoding the
$\omega\times\omega$ tiling problem (see~\cite{BGG97}). Following
the idea in~\cite{BS95}, we construct a spy point over the relation
$s$ which has access to every tile. The relations $u$ and $r$
represent moving up and to the right, respectively, from one tile to
the other. We code each type of tile with a fixed propositional
symbol $t_i$. With this encoding we define for each tiling problem
$T$, a formula $\varphi^T$ such that the set of tiles $T$ tiles
$\omega\times\omega$ iff $\varphi^T$ has a model.


\begin{thm}\label{thm:tlmi:und}
The satisfiability problem for $\cMLRKM + i$ is undecidable.
\end{thm}
\begin{pf}
Let $T=\{T_1,\dots,T_n\}$ be a set of tile types. Given a tile type
$T_i$, $u(T_i)$, $r(T_i)$, $d(T_i)$, $l(T_i)$ will represent the
colors of the up, right, down and left edges of $T_i$ respectively.
Define:

\begin{displaymath}
\begin{array}{rl}
(\textit{Back}) & i \land \bbox{s}\lnot i \land \ttup{s} \top \land \bbox{s}\ttup{s}i \land \bbox{s}\bbox{s}i\\
(\textit{Empty}) & \bbox{s}\lnot \known \land \bbox{s}\bbox{s}(\lnot i \to \lnot \known) \land \bbox{s}[\dag]\lnot \known \quad \dag \in \{r,u\}\\
(\textit{Spy}) & \bbox{s}[\dag]\ttup{s}(i \land \ttup{s}(\known \land \lnot\diam{\dag}\known)) \quad \dag \in \{r,u\}\\
(\textit{Grid}) & \bbox{s}\diam{\dag} \top \quad \dag \in \{r,u\}\\
(\textit{Func}) & \bbox{s}[\dag]\ttup{s}\ttup{s}(\known \land \diam{\dag}\known \land [\dag]\known) \quad \dag \in \{r,u\}\\
(\textit{Conf}) & \bbox{s}\ttup{u}\ttup{r}\ttup{s}\ttup{s}(\known \land \lnot \ttup{r}\known \land \ttup{u}\known \land \ttup{r}( \lnot \known \land (\ttup{u}(\known \land \lnot \ttup{r}\known)))\\
(\textit{UR-no-Cycle}) & \bbox{s}\bbox{u}\bbox{r}\lnot \known \land \bbox{s}\bbox{r}\bbox{u}\lnot \known\\
(\textit{URU-no-Cycle}) & \bbox{s}\bbox{u}\bbox{r}\bbox{u}\lnot \known\\
(\textit{Unique}) & \bbox{s} \left( \bigvee_{1\leq i\leq n} t_i \wedge \bigwedge_{1\leq i < j \leq n } (t_i\to \lnot t_j)\right) \\
(\textit{Vert}) & \bbox{s} \bigwedge_{1\leq i\leq n} \left( t_i \to \ttup{u} \bigvee_{1\leq j\leq n,u(T_i)=d(T_j)}  t_j\right) \\
(\textit{Horiz}) & \bbox{s} \bigwedge_{1\leq i\leq n} \left( t_i \to \ttup{r} \bigvee_{1\leq j\leq n,r(T_i)=l(T_j)}  t_j\right) \\
\end{array}
\end{displaymath}
Let the formula $\varphi^T$ be the conjunction of all the above
formulas. We show that $T$ tiles $\omega\times\omega$ iff
$\varphi^T$ is satisfiable.

Suppose $\model,w\models\varphi^T$. Observe that (\textit{Back}) and
(\textit{Spy}), together with (\textit{Empty}) impose $w$ to be a
spy point over all its $S$-accessible states of $\model$ (and also
force $U$ and $R$ to be irreflexive and asymmetric). These
$S$-accessible states will be the tiles. From this it follows that
$[s]\psi$ holds at $w$ iff $\psi$ is true at every tile.
Additionally, $\diam{s}\diam{s}\psi$ holds at tile $v$ iff $\psi$ is
true at some tile (maybe the same one).

Taking the above points into account, one can establish the
following. (\textit{Grid}) states that from every tile there is
another tile moving up (that is, following the $U$-relation). The
same holds for the right direction (following the $R$-relation).
(\textit{Func}) forces that $U$ and $R$ are both functionals, given
that (\textit{Back}) and (\textit{Spy}) guarantee irreflexivity and
asymmetry of $U$ and $R$ respectively. (\textit{Conf}) imposes that
the tiles are arranged in a grid pattern. To make its job,
(\textit{Conf}) needs the composed relation $U\circ R$ to be
irreflexive, and to avoid a cycle with the $URU$ pattern. This job
is done by (\textit{UR-Irr}) and (\textit{URU-noCycle})
respectively.

All the formulas we discuss up to now configure the grid. The last
three ensure that every tile has a unique type $t_i$, and that the
colors of the tiles match properly. From this, it easily follows
that $\model$ is a tiling of $\omega\times\omega$.

For the converse, suppose $f:\omega\times\omega\to T$ is a tiling of
$\omega\times\omega$. We define the model
$\model=\diam{W,\{S,U,R\},V,\emptyset}$ as follows:
\begin{itemize}
\item $W=\omega\times\omega \cup \{w\}$
\item $S=\{(w,v),(v,w)\mid v\in\omega\times\omega\}$  (hence $w$ will act as the spy
point)
\item $U=\{((x,y),(x,y+1))\mid x,y\in\omega\}$
\item $R=\{((x,y),(x+1,y))\mid x,y\in\omega\}$
\item $V(p)=\{w\}$; $V(t_i)=\{x\mid x\in\omega\times\omega, f(x)=T_i\}$
\end{itemize}
The reader may verify that, by construction,
$\model,w\models\varphi^T$.
\end{pf}

We now turn to the case for $\cMLRKME$.


\begin{thm}\label{thm:tlme:inf}
$\cMLRKME$ lacks the finite model property.
%
%There is a formula $\textit{Inf} \in \cMLRKME$ such that
%$\mathcal{M},w \models \textit{Inf}$ implies that the domain of
%$\mathcal{M}$ is an infinite set.
\end{thm}

\begin{pf}
Consider the following formulas:
$$
\begin{array}{rl}
(\textit{Back}) & p \land \bbox{r} \lnot p  \land \ttup{r} \top \land \bbox{r}\ttup{r} (\known \land p) \\
%&$``The root is irreflexive, and in symmetric relation with all its (nonempty) successors.''$\\
(\textit{Spy}) & \bbox{r}\bbox{r}( \lnot p \to \ttup{r} (\known \land p \land \ttup{r} ( \known \land \lnot \ttup{r} ( \known \land \lnot p))))\\
%&$``A 2-step successor is a 1-step successor, with a non-symmetric relation between them.''$\\
(\textit{Irr}) & \lnot \ttup{r} ( \lnot p \land \ttup{r} ( \lnot p \land \known)))\\
%&$``All successors are irreflexive.''$\\
(\textit{Succ}) & \bbox{r}\ttup{r} \lnot p \\
%&$``All successors have a successor different from the root.''$\\
(\textit{no-3cyc}) & \lnot \ttup{r}   \ttup{r} ( \lnot \known \land \ttup{r} ( \lnot \known \land \ttup{r} ( \known \land \lnot p)))\\
%&$``There are no cycles of 3 elements between successors of the root.''$\\
(\textit{Tran}) & \bbox{r} ( \lnot p \to \ttup{r} ( \known \land p \land \bbox{r} ( \lnot p \land \lnot \known \to \ttup{r}( \known \land p \land \ttup{r} ( \known \land \lnot p \land  \ttup{r}(\known \land \lnot p) )))))\\
%&$``Every pair of successors $u$ and $v$ are related (either $uRv$ or $vRu$).''$\\
\end{array}
$$

Let $\textit{Inf}$ be $\textit{Back} \land \textit{Spy} \land
\textit{Irr} \land \textit{Succ} \land \textit{no-3cyc} \land
\textit{Tran}$. Let $\model=\diam{W, R, V, \emptyset}$. One can
follow the idea used in the proof of Theorem~\ref{thm:tlmi:inf} to
show that if $\model, w \models \textit{Inf}$, then $\model$ is
infinite.

In that proof, the use of $i$ was to identify the spy point. We now
don't have nominals, but using $S=\emptyset$, we can replace $i$
with a clever combination of a propositional symbol $p$ and
$\known$. Since we want to univocally identify the spy point and $p$
may hold in many states, sometimes we need to use $\known$ to
distinguish which one is the spy point (in the proof of
Theorem~\ref{thm:tlmi:inf}, this was not necessary because the
nominal $i$ is true at only one state).

Notice that \textit{Back}, \textit{Spy}, \textit{Succ}, and
\textit{no-3cyc} are very similar to the ones in the proof of
Theorem~\ref{thm:tlmi:inf}, \textit{Irr} forces $R$ to be
irreflexive and \textit{Tran} says that every pair of successors $u$
and $v$ are related (either $uRv$ or $vRu$), and this together with
the other formulas, implies that $R$ is transitive.

%
%Suppose $\model, w \models \textit{Inf}$. Let $B = \{b \in W \mid
%wRb\}$. Because $\textit{Back}$ is satisfied, $w \not \in B$, $B
%\not= \emptyset$ and for all $b \in B$, $bRw$. Because
%$\textit{Spy}$ is satisfied, if $a \not= w$ and $a$ is a successor
%of an element of $B$ then $a$ is also an element of $B$. As
%$\textit{Irr}$ is satisfied at $w$, every state in $B$ is
%irreflexive. As $\textit{Succ}$ is satisfied at $w$, every point in
%$B$ has a successor distinct from $w$. As $\textit{3cyc}$ is
%satisfied, there cannot be $3$ different elements in $B$ forming a
%cycle, and this sentence together with $\textit{Tran}$ force $R$ to
%transitively order $B$.
%
%It follows that $B$ is an unbounded strict partial order, hence
%infinite, and so is $W$.
\end{pf}


In the same way as before, we now show that the satisfiability
problem for $\cMLRKME$ is undecidable, by encoding the $\omega
\times \omega$ tiling problem.

\begin{thm}\label{thm:tlme:und}
The satisfiability problem for $\cMLRKME$ is undecidable.
\end{thm}
\begin{pf}
Let $T=\{T_1,\dots,T_n\}$ be a set of tile types. Given a tile type
$T_i$, $u(T_i)$, $r(T_i)$, $d(T_i)$, $l(T_i)$ will represent the
colors of the up, right, down and left edges of $T_i$ respectively.
Define:

\begin{displaymath}
\begin{array}{rl}
(\textit{Back}) & p \land \bbox{s}\lnot p \land \ttup{s} \top \land \bbox{s}\ttup{s}(\known \land p) \land \bbox{s}\bbox{s}(\known \land p)\\
(\textit{Spy}) & \bbox{s}[\dag](\lnot p \land \ttup{s}(\known \land p \land \ttup{s}(\known \land \lnot\diam{\dag}\known))) \quad \dag \in \{r,u\}\\
(\textit{Grid}) & \bbox{s}\diam{\dag} \top \quad \dag \in \{r,u\}\\
(\textit{Func}) & \bbox{s}[\dag]\ttup{s}\ttup{s}(\known \land \diam{\dag}\known \land [\dag]\known) \quad \dag \in \{r,u\}\\
(\textit{Conf}) & \bbox{s}\ttup{u}\ttup{r}\ttup{s}\ttup{s}(\known \land
\lnot \ttup{r}\known \land \ttup{u}\known \land
\ttup{r}\ttup{u}(\known \land \lnot \ttup{r}\known))\\
(\textit{URU-no-Cycle}) & \bbox{s}\bbox{u}\bbox{r}\lnot \known \land \bbox{s}\bbox{r}\bbox{u}\lnot \known\\
(\textit{URU-no-Cycle}) & \bbox{s}\bbox{u}\bbox{r}\bbox{u}\lnot \known\\
(\textit{Unique}) & \bbox{s} \left( \bigvee_{1\leq i\leq n} t_i \wedge \bigwedge_{1\leq i < j \leq n } (t_i\to \lnot t_j)\right) \\
(\textit{Vert}) & \bbox{s} \bigwedge_{1\leq i\leq n} \left( t_i \to \ttup{u} \bigvee_{1\leq j\leq n,u(T_i)=d(T_j)}  t_j\right) \\
(\textit{Horiz}) & \bbox{s} \bigwedge_{1\leq i\leq n} \left( t_i \to
\ttup{r} \bigvee_{1\leq j\leq n,r(T_i)=l(T_j)}  t_j\right)
\end{array}
\end{displaymath}
Let the formula $\varphi^T$ be the conjunction of all the above
formulas. One can show that $T$ tiles $\omega\times\omega$ iff
$\varphi^T$ is satisfiable, following the ideas used in the proof of
Theorem~\ref{thm:tlmi:und} and the observation in the proof of
Theorem~\ref{thm:tlme:inf}.

%Suppose $\model,w\models\varphi^T$. Observe that (\textit{Back}) and
%(\textit{Spy}) impose $w$ to be a spy point over all its
%$S$-accessible states of $\model$ (and also force $U$ and $R$ to be
%irreflexive and asymmetric). These $S$-accessible states will be the
%tiles. From this it follows that $[S]\psi$ holds at $w$ iff $\psi$
%is true at every tile. Additionally, $\diam{S}\diam{S}\psi$ holds at
%tile $v$ iff $\psi$ is true at some tile (maybe the same one).
%
%Taking the above points into account, one can establish the
%following. (\textit{Grid}) states that from every tile there is
%another tile moving up (that is, following the $U$-relation). The
%same holds for the right direction (following the $R$-relation).
%(\textit{Func}) forces that $U$ and $R$ are both functionals, given
%that (\textit{Back}) and (\textit{Spy}) guarantee irreflexivity and
%asymmetry of $U$ and $R$ respectively. (\textit{Conf}) imposes that
%the tiles are arranged in a grid pattern. To make its job,
%(\textit{Conf}) needs the composed relation $U\circ R$ to be
%irreflexive, and to avoid a cycle with the $URU$ pattern. This job
%is done by (\textit{UR-Irr}) and (\textit{URU-noCycle})
%respectively.
%
%All the formulas we discuss up to now configure the grid. The last
%three ensure that every tile has a unique type $t_i$, and that the
%colors of the tiles match properly. From this, it easily follows
%that $\model$ is a tiling of $\omega\times\omega$.
%
%For the converse, suppose $f:\omega\times\omega\to T$ is a tiling of
%$\omega\times\omega$. We define the model
%$\model=\diam{W,\{S,U,R\},V,\emptyset}$ as follows:
%\begin{itemize}
%\item $W=\omega\times\omega \cup \{w\}$
%\item $S=\{(w,v),(v,w)\mid v\in\omega\times\omega\}$  (hence $w$ will act as the spy
%point)
%\item $U=\{((x,y),(x,y+1))\mid x,y\in\omega\}$
%\item $R=\{((x,y),(x+1,y))\mid x,y\in\omega\}$
%\item $V(p)=\{w\}$; $V(t_i)=\{x\mid x\in\omega\times\omega, f(x)=T_i\}$
%\end{itemize}
%The reader may verify that, by construction,
%$\model,w\models\varphi^T$.
%
\end{pf}



\begin{thm}
$\cMLRKE$ lacks the finite model property and is undecidable.
\end{thm}

\begin{pf}
Straightforward from the effective translation given as a result of $\cMLRKME \leq \cMLRKE$
(Theorem~\ref{thm:four-le-two}) and theorems~\ref{thm:tlme:inf}
and~\ref{thm:tlme:und}.
\end{pf}


To prove failure of the finite model property for the case $\cMLRK$
we first notice that the following lemma is easy to establish (we only
state it for the mono-modal case; a similar result is true in the
multi-modal case).
Failure of the finite model property is then a direct consequence.

\begin{lem}\label{lem:boxes}
Let $\varphi$ be a formula of modal depth $d$. If
{\em $\diam{W,R_r,V,S},w\models$} \linebreak {\em $\left(\bigwedge_{i=0}^d [r]^i \lnot
\known\right) \wedge \varphi$} then $\diam{W,R_r,V,\emptyset},w\models
\varphi$.
\end{lem}

\begin{cor}
{\em $\cMLRK$} lacks the finite model property.
\end{cor}

\begin{pf}
Using Lemma~\ref{lem:boxes}, one can easily see that the formula
$\textit{Inf}\land \left(\bigwedge_{i=0}^4 [r]^i \lnot
\known\right)$, where $\textit{Inf}$ is the one in the proof of
Theorem~\ref{thm:tlme:inf}, forces an infinite model.
\end{pf}



\begin{cor}
The satisfiability problem for {\em $\cMLRK$} is undecidable.
\end{cor}
%
\begin{pf}
Using the idea of Lemma~\ref{lem:boxes} and the formula $\varphi^T$
in the proof of Theorem~\ref{thm:tlme:und}, we can obtain a formula
$\psi$ such that if $\model,w\models\psi$ then $\model$ is a tiling
of $\omega \times \omega$. For the converse, we can build exactly
the same model as in the proof of Theorem~\ref{thm:tlme:und} and
check that it satisfies $\psi$.
\end{pf}
