In this paper we investigate two members of a family of logics that
we called \emph{memory logics}.  These logics were inspired by the
hybrid logic $\hlogic$: the $\down$ operator can be thought of as a
storage command, and our aim is to carry this idea further
investigating different ways in which information can be stored. We
have proved that, in terms of  expressive power, the memory logics
$\tl$ and $\tle$ lay between the basic modal logic $\K$ and the
hybrid logic $\hlogic$.  Unluckily, the reduced expressive power is
not sufficient to ensure good computational behavior: both $\tl$ and
$\tle$ fail to have the finite model property and moreover their
satisfiability problems are undecidable.

Despite the negative result concerning decidability, we believe that
the new perspective we pursue in this paper is appealing.  Clearly,
it opens up the way to many new interesting modal languages (we
discuss some examples in Sect.~\ref{twist}).  As in the case of
modal and hybrid languages, all of them seem to share some common
behavior, and the challenge is now to discover and understand it.

Much work rest to be done. We are currently working on complete
axiomatizations of $\tl$ and $\tle$, and on model theoretic
characterizations. Extending the language with nominals is a natural
step, and then adapting the internalized hybrid tableau
method~\cite{backburn00:_inter} to the new languages is
straightforward.  More interesting is to explore new languages of
the family (like $\texttt{(push)}$, $\texttt{(pop)}$, or
$\texttt{(forget)}$), and interaction between the memory operators
and the modalities.
%$\diam{r}$ and $[r]$ modalities.

For example, if we restrict the class of models to those in which we
are forced to memorize the current state each time we take a step
via the accessibility relation, then the logic turns decidable (even
though it is still strictly more expressive than $\K$).  More
precisely, changing the semantic definition of $\diam{r}$ to be
$$
\begin{array}{rcl}
\diam{W,\rels,V,S}, w \models \diam{r} \varphi & \iff & \exists w' \in W, R_r(w,w') \textrm{ and } \\
& & \diam{W,\rels,V, S \cup \{w\}}, w' \models \varphi
\end{array}
$$
and calling the resulting logic $\tlm$, then $\K < \tlm < \tl$.
Moreover, $\tlm$ has the bounded tree model property: every
satisfiable formula $\varphi$ of $\tlm$ is satisfied in a tree of
size bounded by a computable funcion over the size of $\varphi$.
Hence, the satisfiability problem of $\tlm$ is decidable.

The work presented in this paper is somehow related in spirit with
the work on Dynamic Epistemic Logic and other update
logics~\cite{vanbenthem05,gerbrandy99}, but as we discuss in the
introduction, our inspiration was rooted in a new interpretation of
the $\down$ binder.
